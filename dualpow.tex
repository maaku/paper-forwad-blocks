Changing the proof of work of an existing chain is normally thought of
as something that can only be accomplished by means of a hard-fork.
However it turns out it is possible to soft-fork a proof-of-work
change while still delivering valid blocks to un-upgraded clients.

The essential observation is that it is a soft-fork ammendment of the
consensus rules to make a block subject to two proof-of-work checks,
and that transitioning difficulty from one to the other is naturally
accomplished by a sliding allocation of block reward.  At the
beginning of the transition period the hash rate would remain as
before, as nearly all of the block reward is allocated to the original
proof-of-work miners, but would slowly decrease as dropping rewards
drive miners to wind down operations when profitability limits are
crossed.  Simultaneously miners of the new proof-of-work algorithm
would deploy new mining hardware to capture the increasing amount of
block reward available to them.  At the end of the transition the
block is still subject to two proof-of-work requirements, but the
original proof-of-work is at minimum difficulty.  During this
transition un-upgraded clients will continue to see blocks produced
with ever smaller difficulty, until the difficulty reaches its minimum
value.

In this scenario total-work security would be progressively reduced
for old clients who only see and validate the original proof-of-work
algorithm, but this degradation of security will occur over a period
of years, be visible in reduction of hashrate, and full security could
be restored for any node at any time by upgrading at their
convenience.  Centralizing mining incentives would emerge rather
rapidly as the reward is reduced for the set of old-PoW miners, but we
will show that this has no adverse effects on censorship resistance in
our scheme.  For the new-PoW mining network a
decentralization-inducing growth of subsidy provides some measure of
protection for the duration of the transition period.

As stated so far, there remains the problem of providing the mechanism
by which the sliding block reward is shared between the two sets of
miners in a dual-PoW setup where a single block is subject to both
proof-of-work requirements.  This requires some level of coordination
between both sets of miners, and the more coordination that is
required the less decentralized mining can be.  The solution we adopt
instead is to have separate block chains with loose synchronization of
state, which sidesteps most of the coordination complications
entierly.

\subsection{Merge Mining Is Also a No-Op Proof-of-Work Upgrade}

We do not wish to imply that that this proposal is in any way an
adversarial move against the current set of bitcoin/double-SHA256
miners.  The scheme we develop for performing a proof-of-work upgrade
has other desirable benefits, and the fact that the scheme requires
changing the work function is merely an artifact of how the mechanism
works.  Whether switching to a new proof-of-work algorithm that
invalidates existing ASIC investments is desirable is outside the
scope of what is discussed here.

Requiring a proof-of-work ``change'' a necessary side effect of the
scheme we are developing here.  However, that new proof-of-work
algorithm could be double-SHA256 merge mining, which would be a no-op
proof-of-work ``upgrade'' as all the same double-SHA256 ASIC hardware
in use today would be able to simultaneously mine both, or at least
could be made to do so with a software or firmware upgrade.

Or the new proof-of-work function could be something completely
different, rendering most existing hardware useless once the
difficulty transition is complete.  Either approach is permitted by
this proposal, and we will make no further comment on what this choice
should be, which is entirely orthogonal to the adoption of
forward blocks as a scaling solution.
