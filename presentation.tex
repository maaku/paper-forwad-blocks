\documentclass[10pt]{beamer}

\usetheme[progressbar=frametitle]{metropolis}
\usepackage{appendixnumberbeamer}

\usepackage{booktabs}
\usepackage[scale=2]{ccicons}

\usepackage{pgfplots}
\usepgfplotslibrary{dateplot}

\usepackage{xspace}
\newcommand{\themename}{\textbf{\textsc{metropolis}}\xspace}

\usepackage{graphicx}
\graphicspath{{.}}
\usepackage{epsfig}
\usepackage{mathptmx}
\usepackage{times}
\usepackage{amsmath}
\usepackage{amssymb}
\usepackage{mathtools}
\usepackage[binary-units]{siunitx}
\usepackage{mdwlist}
\usepackage{xfrac}
\usepackage[utf8]{inputenc}
\usepackage{cite}
\usepackage{url}

\DeclareSIUnit\year{yr}
\DeclareSIUnit\weight{We}
\DeclareSIUnit\bitcoin{btc}
\DeclareSIUnit\block{blk}
\DeclarePairedDelimiter{\ceil}{\lceil}{\rceil}

\title{Forward Blocks}
\subtitle{On-chain/settlement capacity increases without the hard-fork}
\date{October 6, 2018}
% \date{}
\author{Mark Friedenbach}
\institute{No organizational affiliation}
% \titlegraphic{\hfill\includegraphics[height=1.5cm]{logo.pdf}}

\begin{document}

\maketitle

%\begin{frame}{Table of Contents}
%  \setbeamertemplate{section in toc}[sections numbered]
%  \tableofcontents[hideallsubsections]
%\end{frame}

% --------------------------------------------------------------------
\section{Introduction}
% --------------------------------------------------------------------

\begin{frame}{Goals}

  \emph{Forward blocks} arose out of considering how a proof-of-work
  change could be accomplished as a soft-fork, combined with
  mechanisms for soft-fork deployment of privacy-enhancing alternative
  ledgers.  It was later discovered that it \textbf{also} provides
  scaling benefits including:

  \begin{itemize}

  \item Improved censorship resistance through sharding.

  \item Direct on-chain scaling up to \num{3584}x (for bitcoin
    specifically).

  \end{itemize}

  It also provides a few miscellaneous other benefits such as a
  linearized block subsidy and the underlying ledger support for
  future chain enhancements such as confidential transactions and
  sidechains.

  \alert{Note}: This talk is a summary of the paper, which has too
  many moving parts to be fully described in a \SI{30}{\minute} talk.

\end{frame}

\begin{frame}{Definitions}

  A \emph{soft fork} is a tightening of the consensus rules such that
  some blocks which were previously valid are now invalid, but no
  previously invalid blocks become valid. Simply: old nodes still
  see the chain advance.

  A \emph{forwards compatible soft fork} is a soft-fork for which
  un-upgraded nodes still receive and process \textbf{all}
  transactions.

  We are specifically interested in forwards compatibility because it
  fits our prior model for the safety of soft forks:

  \begin{itemize}

  \item Non-mandatory upgrades paths.

  \item No flag days beyond which chain access is limited.

  \item An ability for un-upgraded infrastructure to continue working
    during and after the transition period.

  \end{itemize}

\end{frame}

\begin{frame}{A note on centralization risks}

  \emph{Centralization risks} broadly fall into two categories:

  \begin{enumerate}

  \item Increasing the \emph{cost of validation}, or the amount of
    resources (computation, memory, bandwidth) required to initialize
    and maintain a full-node validator so as to be able to transact on
    the network without trusted third parties.

  \suspend{enumerate}

  Cost of validation is proportional to the number of transactions in
  full nodes and the number of blocks for SPV nodes.

  \resume{enumerate}

  \item Reducing \emph{censorship resistance}, which is that property
    which results from any user being able to make a fair attempt at
    mining a block, with the chance of success proportional to their
    share of the hash rate, no matter how large or small.

  \end{enumerate}

  Censorship resistance has a non-linear relationship to the ratio of
  block propagation time to the average block interval.

\end{frame}

% --------------------------------------------------------------------
\section{Dual Proof-of-Work}
% --------------------------------------------------------------------

\begin{frame}{A block subject to 2 proofs-of-work?}

  \textbf{No}.

  \emph{Forward blocks} involves separate chains with separate
  proof-of-work functions, rather than a single block chain with each
  block subject to multiple work requirements.

  But just for the moment, let's use a single block subject to two
  proofs-of-work as an example.

\end{frame}

\begin{frame}{Achieving a transition in difficulty}

  With a block subject to two work functions, the difficulty needs to
  non-disruptively transition from the old proof-of-work to the new.

  The easiest way to do this is have a sliding block reward that
  transitions block reward from going primarily to the solver of the
  old proof-of-work challenge to the new, over of period of time long
  enough as to prevent mining disruption.

  This is expressed by a function $P(t) \in [0,1]$ which represents
  the proportion of the block reward that goes to the new
  proof-of-work miners vs. the old:

  \begin{equation*}
    P(t) = \mathrm{min}(\frac{t}{\SI{3.5}{\year}}, 1)
  \end{equation*}

  At the end of the transition period, in this case \SI{3.5}{\year},
  the new proof-of-work will represent nearly all of the security /
  network hash power, and the old proof-of-work will be at minimum
  difficulty.

\end{frame}

\begin{frame}{New proof-of-work, or merge mining?}

  We are given an opportunity with the deployment of forward blocks to
  change proof-of-work functions.  But we not required to do so---the
  ``new proof of work'' could be double-SHA256 merged mining.

  I do not wish that ``proof-of-work upgrade'' be interpreted as an
  adversarial move against the current set of bitcoin/double-SHA256
  miners.  Rather it is a direct consequence of the design that the
  forward block chain \textbf{requires} a new proof-of-work function.
  A compatible form of salted merge mining is a sufficiently different
  function to work for this purpose.

  Or it could be something entirely new rendering most existing
  hardware useless once the transition is complete.  Either approach
  is permitted by this proposal, and we will make no further comment
  on what this choice should be, which is entirely orthogonal to the
  adoption of forward blocks as a scaling solution.

\end{frame}

% --------------------------------------------------------------------
\section{Raising the Block-Weight Limit}
% --------------------------------------------------------------------

\begin{frame}{Forced hard-forks}

  The commonly discussed ``safe'' mechanism\footnotemark for raising
  the block weight limit is the \emph{forced hard fork}: move
  transactions into a committed extension block with higher aggregate
  limits---and/or any other consensus changes---and then force the old
  blocks to be empty.

  \footnotetext{Forced hard forks are described as safe, but safety is
    relative.  I object to describing a purposeful denial of service
    attack against un-upgraded nodes as ``safe.''}

  Confirming the validation of transactions is only possible by
  upgrading to a client which understands the extension block.
  Un-upgraded nodes are protected from seeing divergent spend
  histories\footnotemark by the fact that the old blocks are kept
  empty.  To restore service, they are required to upgrade.

  \footnotetext{Again, I'd contend that empty blocks should be
    considered divergent with material consequences.  A lightning node
    needs to see its channel closure, for example.}

\end{frame}

\begin{frame}{Forward blocks}

  Forced hard forks break forward compatibility, but note:

  \begin{itemize}

  \item If we ensure that the extension block only violates aggregate,
    block-level consensus rules, then all transactions in the new
    blocks would hypothetically be valid on the old
    chain.\footnote{Except for coinbases and their children.  We have
      a way of dealing with this.}

  \item Instead of forced empty blocks we can have the old
    \emph{compatibility block chain} repeat the same transactions in
    the same order.

  \item By switching from extension blocks to a separate chain with
    loosely coupled state, the \emph{time warp bug} can be exploited
    to lower compatibility block intervals to keep pace with higher
    limits.

  \end{itemize}

  This new chain which determines transaction ordering we call the
  \emph{forward block chain}.\footnote{A reference to \emph{forward
      observers} of mobile infantry units that scout the path ahead.}

\end{frame}

\begin{frame}{Two chains, two separate ways to scale}

  The forward block chain achieves on-chain scaling by increasing its
  per-block aggregate weight limit while maintaining a fixed, long
  duration target inter-block interval.  This achieves the smallest
  possible impact on centralization risks for both full validators and
  SPV nodes.

  The compatibility block chain can only scale by exploiting the time
  warp bug to lower its expected inter-block interval.  Lowering the
  block interval is strongly centralizing on the compatibility chain,
  but this has \textbf{no effect} on censorship resistance because
  transaction ordering is already determined solely by the forward
  block chain.\footnotemark

  \footnotetext{During the transition between proof-of-works, we
    wouldn't want the forward chain to scale so much that the
    compatibility chain becomes centralized before the forward chain
    has enough mining security.  The flexible cap will prevent this.}

\end{frame}

\begin{frame}{Some annoying loose ends...}

  An (incomplete!) list of issues left open by this explanation:

  \begin{itemize}

  \item Many compatibility blocks could be needed to process a single
    forward block, so having the same miner produce these blocks
    unacceptably introduces the notion of work progress.

  \item A coinbase transaction of the forward block is not seen by
    un-upgraded nodes, so it cannot enter the UTXO set.  Both miners
    need to be paid by the compatibility block's coinbase.

  \item Since different miners\footnotemark generate forward and
    compatibility blocks, there needs to be some form of state
    synchronization in order for:

    \footnotetext{The same set of miners if merged mining is used, but
      miner of a specific forward block would not mine the
      corresponding compatibility blocks except by random chance.}

    \begin{itemize}

    \item Compatibility block miners to learn forward block
      transactions; and

    \item Forward block miners to learn the coinbase transactions of
      the compatibility chain, so as to process them into the UTXO
      set.

    \end{itemize}

  \end{itemize}

\end{frame}

% --------------------------------------------------------------------
\section{Loosely Coupled Chain State}
% --------------------------------------------------------------------

\begin{frame}{Cross-chain header commitments}

  Both the forward chain and the compatibility chain commit to the
  headers of the other.  When a header reaches \num{100}
  confirmations, it becomes \emph{locked-in}:

  \begin{itemize}

  \item Any \emph{locked-in} header must reference a valid block.

  \item When the compatibility chain locks in a forward block header,
    the transactions of that block are added to the \emph{transaction
      processing queue} and its coinbase outputs to the \emph{coinbase
      payout queue}.

  \item When the forward chain locks in a compatibility block header,
    the coinbase of that block enters the UTXO set.\footnote{Subject
      to the usual \num{100}-block maturity requirement.}

  \end{itemize}

\end{frame}

\begin{frame}{One coinbase shared by two chains}

  The coinbase of the forward block has outputs that cannot be spent,
  so they are repeated in the compatibility block's coinbase after
  lock-in.

  The forward block miner claims a portion $P$ of the coinbase reward
  of their block.  The remaining $1-P$ value goes into a fund to pay
  compatibility block miners.  The compatibility block miner gets
  \sfrac{1}{k} the current size of the fund.

  Using the compatibility coinbase to synchronize payments based on
  the state of multiple chains is a common pattern we will reuse.

\end{frame}

% --------------------------------------------------------------------
\section{A Flexible Weight Limit}
% --------------------------------------------------------------------

\begin{frame}{Initial parameters of the forward block chain}

  \textbf{Target block interval}: \SI{15}{\minute}

  Increasing the block interval from \SI{10}{\minute} to
  \SI{15}{\minute} achieves a one-shot boost in censorship resistance,
  at the cost of lengthening confirmation times.  For a chain tailored
  to handling settlement transactions, that's a good trade-off.

  \textbf{Initial max block weight}: \SI{6}{\mega\weight}

  With \SI{15}{\min} blocks, this represents the same transaction
  processing rate as the original chain's limit of
  \SI{4}{\mega\weight} blocks every \SI{10}{\minute}.

\end{frame}

\begin{frame}{Growing the forward block chain with a flexcap}

  The forward block miner is allowed to \emph{bias} their work target
  by up to $\pm 25\%$, making their blocks easier or harder to solve.
  In return, the aggregate weight limit of their block is adjusted:

  \begin{align*}
    x &= \frac{T - T_0}{T_0}\\
    x &\in [-0.25, 0.25]\\
    w(x) &= w_0(2x - 4x^2)
  \end{align*}

  Every $L=2016$ forward blocks, two things happen:

  \begin{itemize}

  \item The base limit $w_0$ adjusts to the (possibly gain limited) average of the past $L$ declared weights.

  \item The new-PoW difficulty adjusts up or down as necessary to maintain the \SI{15}{\minute} inter-block interval.

  \end{itemize}

\end{frame}

\begin{frame}{Offset reward as a function of difficulty bias}

  \begin{figure}
    \centering
    \includegraphics[width=0.9\textwidth]{rewardbias.eps}
    \caption{Offset reward as a function of difficulty bias.}
    \label{rewardbiasfig}
  \end{figure}

\end{frame}

\begin{frame}{Time-warping the compatibility chain}

  For compatibility blocks that are \textbf{not} the last block in a
  difficulty adjustment window:

  \begin{enumerate}

  \item If after constructing the block the transaction queue is
    empty, no additional rules apply.

  \item Otherwise, if the next transaction in the queue is non-final
    for reasons of a time-based lock-time or sequence-lock, the block
    timestamp is set to the minimum value necessary to satisfy that
    lock condition.

  \item Otherwise, the block's timestamp \textbf{must} be set to its
    minimum allowed value, which is one more than the median of the
    \num{11} blocks prior.

  \end{enumerate}

\end{frame}

\begin{frame}{Time-warping the compatibility chain (2)}

  For compatibility blocks that \textbf{are} the last block in a
  difficulty adjustment window:

  The \emph{warp factor} is a ratio $Q$ equal to the forward block
  utilization over the original chain's settlement capacity:

  \begin{equation*}
    Q = \frac{w}{\SI{6}{\mega\weight}}
  \end{equation*}

  The Satoshi difficulty adjustment formula is applied in reverse to
  calculate the block timestamp that would cause the adjustment needed
  to handle $Q$ blocks in expectation every \SI{600}{\second}.

\end{frame}

% --------------------------------------------------------------------
\section{A Smooth Subsidy Schedule}
% --------------------------------------------------------------------

\begin{frame}{Eliminate the ``halvening'' with a continuous subsidy curve}

  We have the freedom to set a different subsidy schedule in forward
  blocks, so long as:

  \begin{enumerate}

  \item The cumulative subsidy never exceeds the \SI{21e6}{\bitcoin}
    total monetary base; and

  \item The cumulative subsidy at any given time does not exceed the
    bitcoin subsidy schedule by too large a margin as to cause
    unreasonable delays in the coinbase payment queue.

  \end{enumerate}

  Most notably, we can use this opportunity to smooth out the step
  function used to calculate subsidy, making subsidy a continuous
  function.

\end{frame}

\begin{frame}{Eliminate the ``halvening'' with a continuous subsidy curve}

  \begin{figure}
    \centering
    \includegraphics[width=0.9\textwidth]{subsidycurve.eps}
    \caption{Alternative Subsidy Curve}
    \label{subsidycurve}
  \end{figure}

\end{frame}

% --------------------------------------------------------------------
\section{Multiple Forward Block Chains}
% --------------------------------------------------------------------

\begin{frame}{Sharding: multiple forward block chains}

  A very significant gain in censorship resistance can be had by
  sharding\footnotemark the forward block chain into multiple chains
  with disjoint UTXO sets.

  \footnotetext{Sharding is a term of art borrowed from the database
    field.  The ``sharding'' described here is largely \textbf{not}
    the sharding talked about in crypto currency projects, but it is
    nevertheless the correct term to use.}

  By splitting the chain into $M$ shards, requiring transactions
  source their funds from a single shard only, and using a
  separately-salted work function for each, a shard block will appear
  in expectation every \sfrac{\SI{15}{\minute}}{$M$}, but blocks from
  a given shard chain will be separated by a full \SI{15}{\minute}.
  This achieves an $M$-fold increase in censorship resistance for a
  given aggregate weight across all shards, if activity is evenly
  distributed amongst the shards.

\end{frame}

\begin{frame}{Transferring value between shards}

  \begin{columns}

    \begin{column}{0.5\textwidth}

      \begin{table} \center
        \caption{Shard Identifier Prefixes}
        \label{shardprefix}
        \begin{tabular}{r l l}
          \toprule
          Shard & Prefix & Opcode\\
          \midrule
          1   &  $\mathtt{0x00}$  &  {\tt FALSE}\\
          2   &  $\mathtt{0x4f}$  &  {\tt 1NEGATE}\\
          3   &  $\mathtt{0x51}$  &  {\tt TRUE}\\
          4   &  $\mathtt{0x52}$  &  \num{2}\\
          \multicolumn{3}{c}{$\dots$}\\
          18  &  $\mathtt{0x60}$  &  \num{16}\\
          19  &  $\mathtt{0x61}$  &  {\tt NOP}\\
          20  &  $\mathtt{0x74}$  &  {\tt DEPTH}\\
          21  &  $\mathtt{0xb0}$  &  {\tt NOP1}\\
          22  &  $\mathtt{0xb3}$  &  {\tt NOP4}\\
          \multicolumn{3}{c}{$\dots$}\\
          28  &  $\mathtt{0xb9}$  &  {\tt NOP10}\\
          \bottomrule
        \end{tabular}
      \end{table}

    \end{column}

    \begin{column}{0.5\textwidth}

      A native segwit output can be prefixed with a \SI{1}{\byte}
      value indicating the destination shard identifier.  This makes
      it unspendable on the forward block chain.  Note that {\tt NOP2}
      and {\tt NOP3} are not usable as such on bitcoin.\\~\

      These \emph{prefixed outputs} are added to the coinbase payout
      queue verbatim, and the value is claimed by the compatibility
      block miner and added to the carry-forward balance.

    \end{column}

  \end{columns}

\end{frame}

% --------------------------------------------------------------------
%\section{The High End of Scaling Limits}
% --------------------------------------------------------------------

\begin{frame}{The high end of scaling limits}

  The compatibility chain's block timestamp must tick forward once
  every 6 blocks, which gives a maximum of \num{3600} compatibility
  blocks per \SI{600}{\second} legacy block interval.

  This is closely matched by a \SI{768}{\mega\weight} maximum weight
  on each of \num{28} separate shards with a \SI{900}{\second} block
  interval.

  Together this would allow for \SI{14.336}{\giga\weight} of
  transactions to be processed every \SI{10}{\minute}, which is about
  \num{1}tx/pp/day for everyone currently alive.

  To prevent denial of service from premature growth, it is
  recommended that a gain limiter of $\pm 0.78125\%$ per adjustment
  period be applied, which results in a maximum growth of
  \SI[per-mode=symbol]{\pm 14.5}{\percent\per\year}.

\end{frame}

% --------------------------------------------------------------------
\section{Extension Outputs}
% --------------------------------------------------------------------

\begin{frame}{Generalized ledger transfer mechanism}

  The coinbase payout queue is useful anytime discrete accounting
  systems are used for maintaining a ledger of value within the same
  block chain.  As examples:

  \begin{itemize}

  \item Splitting the block chain into multiple shards, with transfers
    between shards requiring coordination via explicit transfers, as
    already seen;

  \item Obscuring transaction value via confidential transactions
    (with or without mimblewimble kernel support);

  \item Obscuring the transaction graph via support of ring signature
    or zero-knowledge spends; or

  \item Transferring value between multiple sidechains via a two-way
    peg mechanism.

  \end{itemize}

\end{frame}

\begin{frame}{Generalized coinbase maturity mechanism}

  Coinbase payout queues are also useful for any circumstance where
  the value or other detail of an output depends on the circumstances
  of how the enclosing transaction is mined, and therefore a
  maturation process is required to prevent the fungibility risk that
  comes with allowing transactions that can be invalidated with a
  reorg.  Examples from this problem domain include:

  \begin{itemize}

  \item Block reward for forward and compatibility block miners, as
    already seen;

  \item A rebatable fee market where excess fee beyond the clearing
    fee rate is returned to the transaction author; or

  \item Transaction expiry, or other mechanisms by which a transaction
    may become permanently invalid for some reason other than a reorg
    and double-spend.

  \end{itemize}

\end{frame}

% --------------------------------------------------------------------

\begin{frame}[standout]
  Questions?
\end{frame}

\end{document}
