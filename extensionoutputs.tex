So far we've considered a unified coinbase payout queue for the
purpose of coordinating the distribution of block rewards among the
old-PoW and new-PoW miners, and for state symchronization between
shards.  However the mechanism serves as a general solution to the
problem of transferring value between various ledger extensions,
almost without modification.

The coinbase payout queue is useful anytime discrete accounting
systems are used for maintaining a ledger of value within the same
block chain.  As examples:

\begin{itemize}

  \item

    Splitting the block chain into multiple shards, with transfers
    between shards requiring coordination via explicit transfers, as
    already seen;

  \item

    Obscuring transaction value via confidential transactions (with or
    without mimblewimble kernel support);

  \item

    Obscuring the transaction graph via support of ring signature or
    zero-knowledge spends; or

  \item

    Transferring value between multiple sidechains via a two-way peg
    mechanism.

\end{itemize}

Coinbase payout queues are also useful for any circumstance where the
value or other detail of an output depends on the circumstances of how
the enclosing transaction is mined, and therefore a maturation process
is required to prevent the fungibility risk that comes with allowing
transactions that can be invalidated with a reorg.  Examples from this
problem domain include:

\begin{itemize}

  \item

    Block reward for forward and compatibility block miners, as
    already seen;

  \item

    A rebateable fee market where excess fee beyond the clearing fee
    rate is returned to the transaction author; or

  \item

    Transaction expiry, or other mechanisms by which a transaction may
    become permanently invalid for some reason other than a reorg and
    double-spend.

\end{itemize}

Neither of these are meant as an exhaustive list!  There are so many
applications of coinbase payout queues as a maturation process that we
cannot include them all, and the above should be treated as merely a
list of interesting and/or relevant contemporary proposals, some of
which will be elaborated on in the remainder of this talk.

We briefly lay out the mechanism of generalization, first for
segregated ledgers:

\begin{enumerate}

  \item

    Permit the locking up of funds by sending coins to ``anyone can
    spend'' script identifying the destination ledger and endpoint.
    The value is added to a running total identifying the total
    coinage tracked by the ledger, and the funds are immediately
    available at the endpoint, within the segregated environment.

  \item

    The ``anyone can spend'' output paying into the ledger is claimed
    by the miner who creates the block containing the transaction,
    with the coinage added to the carry-forward balance used for
    coinbase payouts.

  \item

    At a later point in time, any owner of funds on the segregated
    ledger can authorize a withdrawal, and in doing so specify the
    destination/recipient.  The amount is subtracted from the running
    balance of funds on the segregated ledger, and an output of the
    specified size and intended recipient is added to the coinbase
    payout queue.

\end{enumerate}

For transaction outputs subject to maturation, the process is even
simpler:

\begin{enumerate}

  \item

    The funds are sent to an ``anyone can spend'' script identifying
    the type of output and intended recipient/destination script.

  \item

    The miner who includes the transaction in their block spends the
    output, adding the funds to the carry-forward balance, and adds an
    equal valued output\footnote{The output has the same value, but
      with a {\tt scriptPubKey} stripped of its extra-protocol
      components.}  to the end of the coinbase payout queue.

\end{enumerate}

The output arrives in the hands of its intended recipient, in the
intended ledger, via the usual process of coinbase maturation.

We will make this abstract process more concrete with a number of
examples drawn from protocol extensions that could be deployed in the
very near future.

\subsection{Rebateable fees and a consensus-visible fee market}

\subsection{Confidential transactions and mimblewimble}

\subsection{Unlinkable anonymous spend ledgers}

\subsection{Sidechains and the two-way peg}
