So far we've considered a unified coinbase payout queue for the
purpose of coordinating the distribution of block rewards among the
old-PoW and new-PoW miners, and for state synchronization between
shards.  However the mechanism serves as a general solution to the
problem of transferring value between various ledger extensions,
almost without modification.

The coinbase payout queue is useful anytime discrete accounting
systems are used for maintaining a ledger of value within the same
block chain.  As examples:

\begin{itemize}

  \item

    Splitting the block chain into multiple shards, with transfers
    between shards requiring coordination via explicit transfers, as
    already seen;

  \item

    Obscuring transaction value via confidential transactions (with or
    without mimblewimble kernel support);

  \item

    Obscuring the transaction graph via support of ring signature or
    zero-knowledge spends; or

  \item

    Transferring value between multiple sidechains via a two-way peg
    mechanism.

\end{itemize}

Coinbase payout queues are also useful for any circumstance where the
value or other detail of an output depends on the circumstances of how
the enclosing transaction is mined, and therefore a maturation process
is required to prevent the fungibility risk that comes with allowing
transactions that can be invalidated with a reorg.  Examples from this
problem domain include:

\begin{itemize}

  \item

    Block reward for forward and compatibility block miners, as
    already seen;

  \item

    A rebatable fee market where excess fee beyond the clearing fee
    rate is returned to the transaction author; or

  \item

    Transaction expiry, or other mechanisms by which a transaction may
    become permanently invalid for some reason other than a reorg and
    double-spend.

\end{itemize}

Neither of these are meant as an exhaustive list!  There are so many
applications of coinbase payout queues as a maturation process that we
cannot include them all, and the above should be treated as merely a
list of interesting and/or relevant contemporary proposals, some of
which will be elaborated on in the remainder of this talk.

We briefly lay out the mechanism of generalization, first for
segregated ledgers:

\begin{enumerate}

  \item

    Permit the locking up of funds by sending coins to ``anyone can
    spend'' script identifying the destination ledger and endpoint.
    The value is added to a running total identifying the total
    coinage tracked by the ledger, and the funds are immediately
    available at the endpoint, within the segregated environment.

  \item

    The ``anyone can spend'' output paying into the ledger is claimed
    by the miner who creates the block containing the transaction,
    with the coinage added to the carry-forward balance used for
    coinbase payouts.

  \item

    At a later point in time, any owner of funds on the segregated
    ledger can authorize a withdrawal, and in doing so specify the
    destination/recipient.  The amount is subtracted from the running
    balance of funds on the segregated ledger, and an output of the
    specified size and intended recipient is added to the coinbase
    payout queue.

\end{enumerate}

For transaction outputs subject to maturation, the process is even
simpler:

\begin{enumerate}

  \item

    The funds are sent to an ``anyone can spend'' script identifying
    the type of output and intended recipient/destination script.

  \item

    The miner who includes the transaction in their block spends the
    output, adding the funds to the carry-forward balance, and adds an
    equal valued output\footnote{The output has the same value, but
      with a {\tt scriptPubKey} stripped of its extra-protocol
      components.}  to the end of the coinbase payout queue.

\end{enumerate}

The output arrives in the hands of its intended recipient, in the
intended ledger, via the usual process of coinbase maturation.

We will make this abstract process more concrete with a number of
examples drawn from protocol extensions that could be deployed in the
very near future.

\subsection{Rebatable Fees and a Consensus-Visible Fee Market} \label{rebatablefee}

In this proposal any user transaction may contain one or more explicit
rebatable fee outputs.  A rebatable fee output has a non-zero explicit
value and a {\tt scriptPubKey} containing a single data-push.  The
data push consists of a weight value serialized as a variable-length
32-bit integer, followed by the fee rebate script verbatim.

The last transaction of a compatibility block is already granted
special semantics: it is the location of the carry-forward balance
which holds aggregated funds for future coinbase payouts.  We now
grant it further special semantics:

\begin{enumerate}

  \item

    Rebatable fee outputs do \textbf{not} enter the UTXO set of the
    forward block (or shard) chain.  They are not spendable by other
    user transactions.

  \item

    The last transaction of a compatibility block \textbf{must} spend
    every rebatable fee output in that block, with the amount added to
    the carry-forward balance.

  \item

    The committed total weight of a forward block must be greater than
    or equal to both (1) the actual weight of the block under known
    consensus rules; and (2) the sum of the serialized weights of all
    rebatable fee outputs in that block.

  \item

    The \emph{clearing fee rate} of a forward block is defined to be
    equal to the sum of all fees (implicit or rebatable) divided by
    the declared weight of the block.

  \item

    For each rebatable fee output an excess value is calculated, equal
    to the value of the output minus its serialized weight times the
    clearing rate.  A consensus-critical dust heuristic is performed
    to see if the excess fee would be considered spendable in the same
    block without lowering the clearing fee rate, and depending on the
    result:

    \begin{enumerate}

      \item

        If the excess value is insufficient to be profitably spent the
        excess is implicitly added to the transaction fees of the
        block, and split between the forward block and compatibility
        block miners as per the current value of $P$.

      \item

        Otherwise the forward block coinbase is required to pay out
        the excess to the serialized {\tt scriptPubKey} encoded in the
        rebatable fee output.

    \end{enumerate}

\end{enumerate}

This achieves the soft-fork compatible rebatable fee \& incentive-safe
fee market described to the bitcoin-dev mailing list\cite{rebatefeeml}
and based off an earlier hard-fork fee market
proposal\cite{rebatefeehf}.

It also shares a lot of the necessary ground work infrastructure for
forward blocks, even if it may seem unrelated, making a rebatable fee
market a very small addition on top of forward blocks.

\subsection{Confidential Transactions and Mimblewimble} \label{confidentialtx}

\emph{Confidential transactions} is a scheme where the explicit {\tt
  nValue} amount of an output is replaced by a Pedersen commitment,
thereby obscuring the value to anyone who does not possess the
blinding factor.  This allows for selective disclosure of transaction
amounts while maintaining bitcoin's non-inflation guarantees.

\emph{Mimblewimble} is an application of confidential transactions in
which the {\tt scriptPubKey} is not used and therefore knowledge of
the blinding factor of an output serves as authorization of a spend.
Removing explicit authorization scripts prevents selective disclosure,
but gained instead are the efficiency and privacy improvements of
transaction cut-through and aggregation.

Bringing confidential transactions and mimblewimble to bitcoin
requires (1) adding the necessary fields---Pedersen commitments, ECDH
half-protocol nonces, and rangeproofs---to the transaction output data
structure; (2) a mechanism for locking up a pool of funds backing the
total value of all confidential outputs; and (3) the validation rules
to enforce the above.

The transaction output data structure is most easily extended by
adding a hash of the extra fields to the end of the {\tt
  scriptPubKey}.  For compatibility with future soft-forks, a general
mechanism developed such that a native segwit output is allowed up to
eight data additional data pushes (although typically only
\numrange{1}{3} are used), together called the \emph{suffix}, with the
last being a required-minimal serialized integer value specifying the
extended output version.

The version corresponding with a confidential transaction output
consists specifies an intervening data value consisting of a single
\SI{32}{\byte} SHA256 hash of the Pedersen commitment and ECDH nonce.
The rangeproof is placed within the output's witness data
structure.\footnote{Outputs need to be given witnesses in most of
  these schemes.  It would be a simple matter to add an \emph{output
    witness structure} whose root hash was also committed somewhere in
  the block, e.g. the ``witness nonce'' of segwit.}  The {\tt nValue}
of the output is \SI{0}{\bitcoin}.

To move coins into a confidential output, the user makes an ``anyone
can spend'' output containing the transferred value and, in an
extended output field, a code indicating this is a transfer into the
confidential transactions / mimblewimble ledger and a Pedersen
commitment of its value (with the necessary signature of the nonce to
prove it in the witness structure).  The output is spent by the
compatibility miner and added to the carry-forward balance, while
within the transaction accounting a confidential ``input'' of the same
amount is added.  The transaction author, being the only one who knows
the nonce of the implicit input, creates confidential outputs claiming
it.

Within the confidential ledger, extended outputs can be spent as
inputs and new outputs created freely so long as the Pedersen
commitments sum correctly in each transaction.

Three mechanisms can be provided by which confidential coins can be
converted back into explicit coins:

\begin{enumerate}

  \item

    An explicit fee can be provided in the form of a ``provably
    unspendable'' output with an extended field specifying how much
    coinage should be added to the output side of the Pedersen
    commitment sum.  The value is treated the same as an implicit fee
    would be.

  \item

    A confidential to explicit conversion can be performed by
    including a ``provably unspendable'' output with extended data
    fields specifying the amount and destination.  Upon achieving
    locked-in status, an explicit output of that amount and to the
    specified destination is added to the coinbase payout queue.

  \item

    A rebatable confidential fee combines both of the above categories
    by providing an explicit amount that is counted against the
    confidential outputs side of the Pedersen commitment sum, and a
    \emph{return script} to which any excess funds should be sent.  It
    is treated as a rebatable fee described in
    Section~\ref{rebatablefee}.\footnote{It would be possible to
      support returning funds as a ``confidential'' Pedersen
      commitment, by providing the EC point corresponding to the
      blinding factor, but (1) such an output would not be
      confidential in any meaningful sense as its value would be
      determined by consensus; and (2) this wouldn't save any block
      chain space over the alternative of having the user aggregate
      return fee excess into confidential outputs themselves.}

\end{enumerate}

Moving from confidential to explicit value requires use of the
coinbase payout queue because the carry-forward balance that is where
the locked up coinage backing the confidential value resides.  If the
user cannot wait for coinbase maturation, they can find another user
willing to trustlessly front the money by signing an explicit input
that covers the amount, minus their service fee, and a confidential to
explicit payout to themselves.

\subsection{Unlinkable Anonymous Spend Ledgers}

\emph{Zcash} demonstrates how a \emph{zk-SNARK} can be used to prove a
spend is from an ever growing set of historical anonymous outputs and
that the output has never been spent before, but without revealing
which output it was.  \emph{Bulletproofs} enable similar constructs
with only elliptic curve discrete log assumptions, albeit with higher
validation cost.  \emph{Monero} achieves a weaker property using
\emph{ring signatures} to prove that an input was one of $N$
explicitly enumerated previous inputs, thereby gaining efficiency at
the cost of a reduced anonymity set.

Any of the above approaches could be deployed in a similar manner to
confidential transactions discussed in Section~\ref{confidentialtx}:

\begin{enumerate}

  \item

    An anonymous output is a zero-{\tt nValue} ``provably
    unspendable'' output containing some sort of commitment to its
    actual value.

  \item

    Anonymous outputs are tracked with their own ledger:

    \begin{enumerate}

      \item

        Creating an anonymous output requires a \emph{transfer in}
        which sends explicit value to an ``anyone can spend'' output
        claimed in a compatibility block and added to the
        carry-forward balance.

      \item

        A ``transfer out'' requires some input from the same ledger
        and specification of the destination, which is added to the
        coinbase payout queue when the transfer out is locked in.

      \item

        Some mechanism exists for paying explicit or rebatable fees
        from the ledger, reducing coins available to outputs of the
        same ledger.

    \end{enumerate}

  \item

    Spending an anonymous output requires keeping some amount of
    information available forever, to prevent double-spends.  Data
    storage on validators can be avoided by using a Merkle tree
    updated on each spend.

  \item

    The actual spend is not an input, as that would require specifying
    an input which defeats the purpose.  Instead, it is a zero-{\tt
      nValue} ``output'' whose witness provides the spend
    authorization, and the committed value pulled from the output
    witness is added on the input side of the anonymous spend ledger.

\end{enumerate}

Which anonymous spend mechanism to use and the format of its spend
authorization witness we leave up to debate; our concern here is with
the block chain ledger accounting and ledger transfer mechanisms.

\subsection{Sidechains and the two-way peg}

For this topic we avoid the issue of which validator-enforced
sidechain architecture should be used---whether the SPV peg,
Drivechains, or something else.  Regardless of the transfer
authorization scheme the accounting mechanism is the same:

\begin{enumerate}

\item

  A \emph{sidechain transfer} is an ``anyone can spend'' output which
  is claimed by the compatibility block miner and added to the
  carry-forward balance, and which commits to the sidechain identifier
  and destination script to receive the funds on the other chain.

\item

  A \emph{return-peg transfer} is an ``provably unspendable'' output
  which commits to the amount and destination of the funds, and whose
  witness provides the necessary sidechain information to validate the
  return peg.  Once locked-in, the return peg is added to the coinbase
  payout queue.

\end{enumerate}

By now it should be clear that all of these extended-ledger mechanisms
share the same common approach, using the carry-forward balance and
coinbase payout queue to manage transfers in and out, and extra data
pushes in the {\tt scriptPubKey} to store extended transaction output
fields.
