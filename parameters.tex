In this section we lay out the suggested initial settings and growth
limits for sharding and the per-shard flexible weight limit in
bitcoin:

\begin{enumerate}

  \item

    The forward block chain is split into 28 shards with separate UTXO
    sets, for a \num{28}x increase in censorship resistance if fully
    utilized.

  \item

    All forward shard chains have a fixed target inter-block interval
    of \SI{15}{\minute}, for an additional few percentage points of
    censorship resistance.

  \item

    At the point of activation, the initial maximum weight per block
    $w_0$ of the first/default forward shard chain is
    \SI{6}{\mega\weight}, and \SI{25}{\kilo\weight} each for the other
    \num{27} remaining shards.

  \item

    Simultaneously with activation, a previously implicit maximum
    transaction weight is imposed with a limit value very close to the
    original chain's maximum weight limit: \SI{3.975d6}{\weight}.  The
    presently lower policy limit still applies, but transactions
    larger than \SI{3.975}{\mega\weight} are now invalid, not just
    non-standard.

\suspend{enumerate}
    
The per-shard target inter-block interval of \SI{15}{\minute} implies
a new block from a randomly selected shard chain arriving every
$\sim\SI{32}{\second}$ or so, which provides a total aggregate
transaction processing rate of \SI{4.45}{\mega\weight} every
\SI{10}{\minute}, only $11.25\%$ more than the original maximum block
weight.

This configuration of the flexible block weight limit sets an initial
limit on the first shard chain that provides a transaction processing
throughput equivalent to the original chain, and therefore starts with
exactly the same validation costs as all UTXO exist in that shard
initially, and with approximately the same censorship resistance.  The
other shard chains provide an additional $11.25\%$ of block space, but
user adoption of this space is likely to take time as wallets and
other infrastructure are adapted, just as was the case with segwit.

Once sharding is fully utilized, a censorship resistance will be
improved by a factor equivalent to having reduced the original max
block weight by $\sim 30$x.

\resume{enumerate}
  \item

    Using a flexible cap, the miner is allowed to grow or shrink by up
    to $\pm 25\%$ the maximum weight of their own forward shard
    blocks, receiving less (when growing) or more (when shrinking)
    block reward as a result, a tradeoff they would only make in
    response to abnormally high or low clearing fee rates.

  \item

    Every \num{2016} blocks (a \textbf{three} week period), each shard
    chain will add to its baseline maximum weight $3.125\% (=
    \sfrac{1}{32})$ of the difference from the previous baseline limit
    to the trailing average of the actual last \num{2016} block's
    weight limits, as were adjusted by the flex cap, constrained to be
    no smaller than \SI{25}{\kilo\weight} and no larger than
    \SI{768}{\mega\weight}.  Thus an adjustment of no more than
    \num{1.0078125}x in either direction is possible per adjustment
    period, resulting in a maximum adjustment of around
    \SI[per-mode=symbol]{\pm 18.4}{\percent\per\year}.

\end{enumerate}

Should every shard chain adjust in response to real usage to its
maximum allowed size of \SI{768}{\mega\weight}, the total transaction
processing throughput on the forward block shard chains taken together
would be $28 \times \SI{768}{\mega\weight}=\SI{21.504}{\giga\weight}$
every \SI{15}{\min}, or expressed as a more conventional rate
\SI{14.336}{\giga\weight} every \SI{10}{\min} which very nearly equals
the maximum allowed throughput on the compatibility chain when fully
exploiting the time-warp feature.  However it would take consistent
maximum upwards adjustments every period for half a century to reach
this limit, and given the quadratic costs to raising the flexcap,
probably much longer.

\subsection{How Reasonable Is This Upper Bound, Really?}

It is not inconceivable that ongoing network and validator
improvements will permit a growth in the decades and centuries to come
of the order of magnitude required, \numrange{7.5}{12} doublings of
compute resources, storage, and bandwidth to safely reach the maximum
throughput of \SI{14.336}{\giga\weight} every \SI{10}{\min}, assuming
today's \SI{4}{\mega\weight} limit as the present-day ``safe''
baseline.  It is worth noting that even in this extreme the
corresponding per-shard weight limits would be less than the maximum
limits allowed by conservative block size increase proposals such as
BIP-103 over the same timeframes.

\SI{768}{\mega\weight} per \SI{15}{\minute} shard-block is ``only''
\num{192}x the post-segwit weight limit of \SI{4}{\mega\weight} per
\SI{10}{\minute} block---about \num{7.5} doublings.  Hitting this
limit would require new 4MWe compatibility blocks generated every
\SI{167}{\milli\second}, fed from a queue of transactions generated by
\num{28} loosely coupled shards, for a total of
\SI{21.504}{\giga\weight} every \SI{900}{\second} / \SI{15}{\minute}.
Staying up to date with such a block chain would require
\SI{240}{Mbps} downlink, and participating as a relay node would
require an approximate \num{4}x multiple thereof, or around
\SI{1}{Gbps}.\footnote{\num{1} download, and \num{3} uploads per
  transaction.}  Costs to validators would be \num{3584}x present-day
costs, or just under \num{12} doublings.

While a lot, and certainly outside of current capabilities, this is
not so crazily outside the realm of possibility as it might seem when
we consider what generations of technologies might exist into the
foreseeable future:

\begin{itemize}
  \item

    Mature graphene transistors, among many possible technological
    pathways, could provide consumer electronics with many orders of
    magnitude more performance for the same power and heat
    dissipation, allowing continued increase in the computational
    capacity available within consumer form factors (e.g. laptops \&
    desktop workstations), and make possible large storage density
    non-volatile memory to remove I/O bottlenecks.

  \item

    Better electro-optical interconnects would allow existing deployed
    fiber optic cables to carry orders of magnitude more information,
    allowing for higher last-mile bandwidth and transfer caps from
    consumer internet services, allowing both wired connections and
    WiFi be linked with \SI{1}{Gbps} or faster WAN connections.

  \item

    Mobile data is unlikely to advance as far (due to spectrum
    scarcity), but the broadcast nature of block chain data means that
    the partially-trusted solution of satellite or terrestrial radio
    broadcast could be used for block distribution to fully validating
    block-only mobile clients.  \SI{240}{Mbps} for absolutely full
    blocks is only about \num{12} HDTV channels, meaning there is
    sufficient spectrum available today for multiple operators to
    redundantly broadcast block data of that size.  Mobile clients
    would only require a lower bandwidth bidirectional connection (3G
    or better) to broadcast their transactions.

\end{itemize}

Looking at the flip side of damand, $28 \times \SI{768}{\mega\weight}
= \SI{21.504}{\giga\weight} / \SI{15}{\min}$ is
\SI{2.064}{\tera\weight} per day in transaction processing capability,
or about \numrange{1}{2}\si[per-mode=symbol]{\tera\byte\per\day} of
block chain data.  Although a lot of data by any reasonable
measurement, it's still only about \num{1} transaction per day for
every human being presently alive, to say nothing of future
populations in the cis-lunar light sphere.  For payments this would
seem to be more than enough as only rebalancing transactions of
payment channels are needed at a frequency of a month or so.  But:

\begin{itemize}
  \item

    The napkin-math calculation of
    $\sim1\mathrm{tx}/\mathrm{day}/\mathrm{pp}$ is an average, whereas
    the need to rebalance less frequently is a statement of the
    median.  We should expect the need for block chain access to
    follow a power law distribution, which very well could push the
    necessary average much higher than the median value, as the median
    is more reflective of the long tail of casual use than the peak of
    businesses and services that make much higher use of the block
    chain as a settlement platform.

  \item

    Focusing on payments is myopic as what bitcoin provides is the
    first, and so far only platform for fully automated trustless
    dispute resolution.  Payments are only the simplest form of what
    we might call \emph{bitcoin-compatible smart contracts} that use
    bitcoin or other crypto tokens as collateral and UTXOs as
    semaphore primitives in the coordination of the actions of
    multiple real-world parties with respect to some pre-arranged
    contract.  While it is possible that multiple contracts can be
    handled with a single pot of funds managed by a multi-party
    off-chain protocol---as is the case with lightning---there is an
    efficiency tradeoff to be made and access to the chain is
    eventually required, always.

\end{itemize}

In this light, it is entirely possible that
$1\mathrm{tx}/\mathrm{day}/\mathrm{pp}$ is below the needs of what
true global adoption would require for a block chain that handles all
forms of automated commercial dispute resolution, not just consumer
payments.  However it is the limit of what can be achieved as a fully
forwards compatible soft-fork in which all nodes see all transactions.

We contend then that these maximum limits are within the realm of
reasonable demand in the circumstance of total world adoption, while
also within the limits of what foreseeable technology might enable
within the decades it would take, even optimistically, to get there.
In the mean time an appropriately configured flexible cap ensures that
a lower limit is maintained until such time in the future as there is
real, paid-for, non-transient demand for that much block space,
ensuring in the process that block size increases cannot be used as a
way of avoiding transaction fees.  Furthermore, a reasonable flex cap
proposal, as we have laid out here, would make sure that increases
happen slowly enough that collective user action could impose a lower
limit if it were seen as being being prematurely raised.
