Rather than have a single block subject to multiple proof-of-work
checks, and all the complications that would arise from that, we
instead propose using a divergent fork of the original chain where the
old-PoW miners look to the new-PoW blocks for ordering of
transactions, and thus a consistent ledger is maintained.  This new
chain which forces the ordering of transactions is called the
\emph{forward block chain} and we will now build up the core concept
in parts.

\subsection{Using a ``Forced Hard-Fork'' To Change Work Functions}

To begin with, consider a hard-fork proof-of-work upgrade, with no
replay protections: at a specified activation height the new set of
miners fork off from the historical chain, creating blocks conforming
to the new proof-of-work requirement, and carry over the same set of
transactions in their mempool for inclusion in future blocks.  The
rules for transaction validation remain the identical to the original
chain, but any block-level consensus rules---such as aggregate
limits---are free to be changed without regard for compatibility with
existing clients, as forward compatibility is necessarily broken
already.

Now combine this with what has been variously called ``forced
hard-forks,'' ``soft hard-forks,'' or ``evil forks'' [\emph{sic}]: a
majority of miners on the original chain at the time of the split
refuse to build upon any transaction-including blocks.  This is how
the transition begins: a hard-fork chain that is continuing to mine
compatible transactions but using a new-PoW function, and the original
chain mining empty blocks with valid old-PoW headers but devoid of any
user transactions.

To make sure that blocks continue to be mined on both chains, the
block reward must be made to slowly transition from the chain
continuing the old-PoW to the new at a rate not much greater than
normal variance in mining income.  Old-PoW miners would witness their
income slowly dropping, while new-PoW miners start receiving
equivalent increases in their block rewards.  Over time less efficient
old-PoW miners will wind down their mining operations causing a
ramping down of difficulty in proportion to the reduction in block
reward.  At the same time, investments in new-PoW mining hardware to
capture the growing share of block rewards on that chain will drive
its mining difficulty (and security) up.

As the name implies, the \emph{forced hard-fork} scenario was designed
to force old nodes to upgrade to newer software versions which
understand the new proof-of-work/extension blocks where transactions
actually reside, as otherwise they are unable to witness any
transaction confirm.  It is considered by some to be ``safe'' because
old nodes cannot be tricked into seeing a differing spend history on
the old chain compared to the new.  But sometimes not being aware of
transactions can be just as dangerous as seeing the wrong
transactions---e.g. not seeing a channel closure from old state---and
it is coercive in requiring an upgrade.  In the next two subsections
we will explore how we can restore transaction processing on the
old-PoW chain without risking a divergent history.

\subsection{Preventing Double Spends with a Transaction Queue} \label{transactionqueue}

To restore transaction processing for un-upgraded clients we alter
this ``forced hard-fork'' setup in the following ways:

\begin{enumerate}
  \item
    The coinbase outputs of the new-PoW chain are \textbf{not} entered
    into the UTXO set.  They serve a purpose that is explained in
    Section~\ref{coinbasepayoutqueue}, but they are not themselves
    spendable in either chain.

  \item The coinbase outputs of the old-PoW chain enter the UTXO set
    of \textbf{both} chains, after a maturation process explained in
    Section~\ref{coinbaseconfirmation}.  To be clear, this means a
    post-fork coinbase output of the old-PoW chain will \textbf{also}
    be spendable on the new-PoW chain after the maturation process
    following synchronization of state between them.
\suspend{enumerate}

With these changes a transaction valid on one chain is also valid on
the other, and this remains true indefinitely into the future, so long
as a consistent ordering of transactions is used in both chains.

At this point in the explanation a double spend is trivially not
possible because the old-PoW chain lacks any transactions at all,
other than its coinbase, so UTXO set compatibility may seem an
academic concern.  But we now relax this restriction without enabling
transaction history to diverge:

\resume{enumerate}
  \item
    The coinbase of the old-PoW chain is allowed to commit to block
    headers of the new chain, if doing so extends the chain of known
    headers to a more-work tip of equal or greater height than was
    previously known from prior commitments.

  \item
    If at any time a header is buried by $N$ or more further block
    headers in the same chain, then the buried header
    is \emph{locked-in} and cannot be subject to a later header
    commitment reorg, except by reorg'ing the old-PoW chain to remove
    the $N$th commitment.

  \item
    Should an attempt be made to commit to a chain of block headers
    that would reorg an already locked-in header or even just remove
    its locked-in / $N$th confirmation status, the block containing
    that commitment is invalid.

  \item
    Adding an $N$th confirmation of a block requires that buried and
    now locked-in header reference a valid block.
\suspend{enumerate}

Thus full-block validation of the other chain is allowed to happen
asynchronously during the maturation period, while any block
referenced by a locked-in header \textbf{must} be valid.  It also
means that a block header commitment could be to an invalid block, but
only so long as that block header never receives a $N$th
confirmation---or else the block containing that $N$th confirmation
commitment is invalid.

\resume{enumerate}
  \item
    Once the old-PoW chain has accumulated $N$ confirmations of a new
    chain block header, resulting in that new-chain block acquiring
    ``locked-in'' status, the transactions in that block are added to
    the end of an first in, first out transaction queue.  This queue
    is initially empty at the point of activation of the fork, and
    transactions are added to the queue after creation/validation of
    the block containing the $N$th confirmation commitment, so the
    earliest a transaction can appear is in the following block.

  \item
    When the transaction queue is non-empty, and so long as the queue
    remains not-empty, new blocks in the original chain are required
    to include transactions from that queue, in order, for as long as
    the next transaction in the queue is valid and including it would
    not violate aggregate limits for the block.
\end{enumerate}

A transaction will only ever be invalid for reasons of finality
(time-locks or sequence locks) or coinbase maturity, both being
infrequent edge cases\footnotemark that resolve with time, after which
blocks contain transactions again, pulling from the queue where the
transaction processing stopped.

\footnotetext{It is possible for a transaction to be temporarily
  invalid because finality and coinbase maturity requirements are
  checked differently on both chains.  As a concrete example, a
  transaction with a \num{144}-block relative lock-time must be
  separated from its input by that many blocks on \emph{both} both the
  original chain and the post-fork chain.  If the transaction were to
  first confirm exactly \num{144} blocks after its input on the
  new-PoW chain, but one of the interleaving blocks is empty, then the
  old-PoW chain could process the interleaving transactions in
  only \num{143} blocks.  Doing so however would violate
  the \num{144}-block relative lock-time.}

Observe that we have now restored transaction processing on the old
chain without permitting divergent history, since the selection and
ordering of transactions on the old chain is fully determined by the
lock-in of block header commitments to the new chain, itself an
irreversible process within the context of a single chain history.  A
large reorg of the new chain can still be handled, but at the cost of
requiring a reorg of the original chain to a point before the $N$th
confirmation of the point of divergence.

Simply put, the miners of the new chain provide a deterministic order
of transactions which \textbf{must} be followed by in the continuation
of the original chain after the point of activation.  The old-PoW
miners individually lose their control over transaction selection, as
that is now fully determined by the most-confirmed-work of the new-PoW
chain.\footnote{Although when merged mining is used, the old-PoW
miners are also the new-PoW miners.}  To prevent divergent UTXO sets
the coinbase outputs of the old chain are used to collect fees, pay
out miners of both chains, and other purposes requiring coinbase
maturation, by a process explained in
Section~\ref{coinbaseconfirmation} and
Section~\ref{coinbasepayoutqueue}.

The purpose of the new-PoW chain is to provide a transaction ordering
which \textbf{must} be used by the old-PoW miners in the construction
of their blocks.  For this reason a new-PoW block called a
\emph{forward block} and the new chain the \emph{forward block chain}
in analogy to the concept in finance and real estate of a ``forward
transaction''---the purchase of a good, service, or property
\textbf{now} at a certain price for delivery on a fixed future
date---or the ``forward observer'' of an infantry battalion which
scouts in advance the route taken by a unit on the move.  The miners
of the new chain are choosing which transactions and in what order the
old-PoW miners are required to include in their chain, after $N$
confirmations of the block headers has been achieved in the block
header commitments.

The old-PoW block is called a \emph{compatibility block} and the
original chain the \emph{compatibility chain} because it relays, after
some delay, transactions in the already agreed upon order to
non-upgraded clients, and processes coinbase payouts in a way that
would be seen and identically processed by old and new clients alike.

\subsection{Coinbase Confirmation on the Forward Block Chain} \label{coinbaseconfirmation}

We are now ready to explain how compatibility chain coinbases enter
the UTXO set of the forward block chain, without which there would be
no way for miners of either chain or recipients of coinbase payouts in
general to collect their due.  The mechanism for making coinbases
spendable is a similar set of block header commitments and maturity
rules as was used in Section~\ref{transactionqueue} to construct the
transaction queue on the compatibility chain, this time on the forward
block chain and with regard to coinbase payouts:

\begin{enumerate}
  \item
    The coinbase of forward blocks are allowed to commit to block
    headers of the compatibility chain, if doing so extends the chain
    of known headers to a more-work tip than was previously known from
    prior commitments.  A reorg of no more than $N$ block headers is
    allowed in the process.\footnote{Technically the two separate $N$
    parameters used in the compatibility-block and forward-block
    settlement periods could be different, but there is no security
    benefit for that to be so, and doing so would only add delays to
    the maturation process.}  Only block headers are validated until
    the $N$th confirmation, at which point block validity is also
    required.

  \item
    Once the forward block chain has accumulated $N$ confirmations of
    a compatibility block header, that block's coinbase outputs are
    entered into the UTXO set with an $M$-block maturity (allowing
    them to be spent in $M$ additional blocks after the inclusion of
    the $N$-th block header commitment).
\end{enumerate}

$N$ is a free protocol choice known as the \emph{forward-block
  settlement period}, and $M$ is the \emph{coinbase maturity period},
  both of which combine to define the total length of time it takes
  for a coinbase payout to mature.  We use $N = 96$ and $M = 100$ in
  our implementation.\footnote{In Section~\ref{parameters} we
  recommend \SI{15}{\min} forward block intervals, which means $N =
  96$ blocks is approximately \num{1} day.}  Having $M$ greater than
  $N$ doesn't provide any higher level of security, but has the
  advantage of not requiring alteration of any
  existing \num{100}-block coinbase maturity handling code, at the
  cost of waiting only an extra \num{4} blocks.

We can now construct a complete timeline for what happens when the new
rules are activated:

\begin{enumerate}
  \item
    At the time of activation of the fork the forward block chain
    splits from the compatibility chain, and the forward block miners
    begin minting blocks using the new-PoW and transactions selected
    from their mempools, using possibly different aggregate limits and
    other block-level consensus rule changes.  The non-coinbase
    transactions included in the forward block chain would be valid on
    the original chain, in the counter-factual scenario that the new
    rules had not activated (assuming the equivalent block height and
    block time for time-locks, the equivalent ordering and separation
    for sequence locks, etc.).

  \item
    Simultaneously the upgraded original-PoW miners, constituting a
    supermajority of hash power at the time of activation, begin
    mining blocks for the compatibility chain, initially consisting of
    only a coinbase transaction which may have zero, one, or more
    commitments to new-PoW forward block header(s).

  \item
    As blocks progress on the compatibility chain, forward block
    miners include reciprocal commitments in the coinbases of their
    own forward blocks to the advancing tip of the old-PoW
    compatibility chain.

  \item
    When in either chain knowledge of the other exceeds $N$ blocks
    past the point of fork, asymmetrical effects occur:

    \begin{enumerate}
      \item
        Once the compatibility chain has added knowledge of a forward
        chain tip $N$ or more blocks beyond the point of fork,
        transactions other than the coinbase in forward blocks with
        $N$ or more known confirmations are added to the transaction
        queue in the same order as they were seen in the forward block
        chain.  Starting with the very next block, compatibility chain
        miners now reject any compatibility block which does not draw
        from the transaction queue filled by prior blocks, in the
        order added, allowing for less-than-full blocks only when the
        next transaction in the queue breaks a finality or maturation
        rule.

      \item
        Once the forward chain has added knowledge of a compatibility
        chain tip $N$ or more blocks beyond the point of fork, the
        coinbase outputs of compatibility blocks with $N$ or more
        known confirmations are added to the UTXO set and become
        spendable after an $M$ block maturity period.
    \end{enumerate}
\end{enumerate}

For an upgraded wallet, confirmations on the forward block chain are
the only confirmations that matter, as later confirmation on the
compatibility chain is predetermined.  For a non-upgraded wallet there
is a perceptible and persistent delay in confirmation on the
original/compatibility chain that they see, but transactions do
eventually confirm and progress is made.  So long as applications are
written to gracefully handle confirmation delays, applications running
against pre-fork versions of the client software will continue to work
during the transition.
