A necessary condition for transaction compatibility across the forward
block and compatibility chains is that the forward block coinbase
outputs \textbf{not} enter the UTXO set, as they are not visible to
un-upgraded nodes following the compatibility chain, and any
transactions spending such outputs would be considered invalid by
those nodes.  However it is necessary that forward block miners be
able to collect subsidy and fees, and generally be paid for the work
that they do, which is a hard thing to accomplish when the outputs
they have control over are unspendable.

To solve this problem we provide a mechanism for delayed cross-chain
state synchronization, and use the block reward on the compatibility
chain to pay out both sets of miners.  In doing so we develop a
mechanism for synchronizing state across chains or separate ledgers
generally which we will reuse in our discussions of sharding in
Section~\ref{sharding} and potential future privacy-enhancing
extensions in Section~\ref{extensionoutputs}.  However the best way to
introduce this mechanism is through the one other solution it is
designed to solve: providing a smooth difficulty transition between
the old an new proof-of-work functions which we will cover in
Section~\ref{smoothdifficulty}.  But first we cover in this section
the basic underlying mechanisms involved.

Forward compatibility demands that all block reward (subsidy and fees)
be processed in the compatibility block chain.  However payouts for
both groups of miners are unconnected to whatever block reward happens
to be available to a specific compatibility block.  We resolve this
issue by aggregating block rewards into a common fund that is used to
payout miners on a first-in, first-out basis.  Specifically:

\begin{enumerate}
  \item
    Forward block miners are allowed a portion of the subsidy and fees
    of the blocks they mine.  They specify the desired destination for
    these funds in the dummy outputs of their coinbase transactions.

  \item
    Compatibility block miners are alloted a portion of the remaining
    block reward from already locked-in forward blocks.  They too
    specify the desired destination for their share of the coinbase
    reward in a special commitment in their blocks.
\suspend{enumerate}

The mechanism for dividing block reward between the forward and
compatibility miners is the topic of Section~\ref{smoothdifficulty}.
Suffice to say there is a fair mechanism for determining how much of
the subsidy and fees are allocated to each set of miners, and the
reward made available to compatibility chain miners---which depends on
the stochastic forward block finding rate---is smoothed out.

When a compatibility block is processed, the desired outputs of the
compatibility block miner are added to a first-in, first-out
\emph{coinbase payout queue}.  The coinbase outputs of any forward
block that achieves its $N$th confirmation and locked-in status is
also added to the end of this queue.

The coinbase of the compatibility block is required to pay out the
collected subsidy and fees to outputs pulled from the payout queue in
order.  If there is insufficient reward remaining to process the next
coinbase payout, or if the payout queue is empty, the remaining block
reward is carried forward to cover future payouts.  Procedurally:

\resume{enumerate}
  \item
    The compatibility block miner commits to their own outputs, of sum
    total equal to or less than the output of the compatibility block
    reward function $R$ specified in Section~\ref{smoothdifficulty}.
    The committed outputs are added to a payout queue maintained by
    each validator, before processing any other miner commitments.

  \item
    The compatibility block miner also commits to a possibly
    zero-length chain of headers extending the known tip of the
    forward block chain.  When a forward block receives its $N$th
    confirmation reported to the compatibility chain, the forward
    block's coinbase outputs are also added to the same payout queue.

  \item
    The normal block reward that would have been available for the
    old-PoW miners if (counter-factually) the new rules hadn't
    activated, plus any carry-over from the prior block, is used to
    fund outputs from the queue in the order added.  If at any point
    there is insufficient funds to pay out the next output or if the
    payout queue is empty, the remaining block reward is put into a
    special \emph{carry-forward output}.

  \item
    Once the compatibility block receives $N$ confirmations in block
    headers reported to the forward block chain, all the outputs of
    its coinbase except for the carry-forward output are entered into
    the UTXO set of the forward block chain, with an $M$ block
    maturity period.

  \item
    Once the carry-forward output matures according to the
    compatibility block rules, it is spent by a special
    miner-generated transaction, the last transaction of the block in
    which it became spendable.  If there are still payouts to be made
    from the payout queue, the necessary amount of funds are released
    as ``fee'' to be collected in the coinbase and paid out, otherwise
    the remaining funds are placed in a carry-forward output of this
    transaction to process coinbase payouts of the next block, or as
    needed in the future.
\suspend{enumerate}

Compatibility chain miners are required to process outputs (meaning,
include the payout in the block's coinbase transaction) taken from the
payout queue, in order, until one of three stopping conditions are
reached: (1) the payout queue is exhausted; (2) the block reward and
carry-forward balance combined is less than the next item in the
payout queue; or (3) adding additional coinbase outputs would exceed
aggregate block limits.  Remaining funds from the block reward and
carry-forward balance are placed in the coinbase and last-transaction
carry-forward outputs, respectfully.

One final rule settles contention between use of block space for
making coinbase payouts and processing transactions:

\resume{enumerate}
  \item
    The coinbase payout queue is processed before miner commitments or
    the transaction queue, so the contribution of the coinbase towards
    aggregate block limits is known.
\end{enumerate}

The above rules are consensus enforced---blocks are considered invalid
if they do not adhere to all of the above rules.

Now we have the right pieces in place to explain how mining rewards
are claimed by both sets of miners:

\begin{enumerate}
  \item
    A forward block is mined, containing a fake coinbase transaction
    for the purpose of specifying where the forward block miner wants
    their share of the block reward to go, as well as various other
    required or optional commitments.

  \item
    Since the compatibility block reward function $R$ is a function of
    the locked-in forward blocks over a specific period and there is
    very little cost to including a forward block header commitment,
    compatibility miners have incentive to include any valid forward
    block headers they know about in their block templates.

  \item
    When a compatibility block is found, its coinbase contains payouts
    pulled from the front of the queue, and its own payouts are added
    to the end of the queue---both the compatibility miner's payout
    preferences and the outputs of any newly locked-in forward blocks.

  \item
    Desiring to be paid for their prior work, the forward block miners
    have incentive to include commitments to the longest compatibility
    block chain they know about in their own block templates.

  \item
    When a forward block is found, the coinbase transactions of any
    newly locked-in compatibility blocks enter the UTXO set, subject
    to the usual \num{100} block maturation delay.
\end{enumerate}

By this mechanism both forward and compatibility block miners receive
their pay by adding commitments to each other's blocks, maintaining
the loosely coupled connection between the two chains.
