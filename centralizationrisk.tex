Centralization risks in bitcoin are generally concerning for two
reasons: increasing \emph{cost of validation} and decreasing
\emph{censorship resistance}.

The \emph{cost of validation} is the amount of resources (computation,
memory, bandwidth) required to initialize and maintain a full-node
validator so as to be able to transact on the network without trusted
third parties.  Any increase of the on-chain settlement throughput
necessarily incurs an increase in the cost of validation that is at
least linear.  In terms of magnitude, opinions differ but we contend
that it would be largely noncontroversial to require that full node
validation should require no more than the computational resources of
a recent consumer laptop or desktop workstation, and have bandwidth
requirements met by the lesser of a tail-weighted average of bandwidth
available to home internet users or across onion routing overlay
networks with large anonymity sets.

\emph{Censorship resistance} is that property which results from the
ability of any user to make a fair attempt at mining a block, with the
chance of success proportional to their share of the hash rate, no
matter how small; censorship being when another entity is able to
unfairly control one's access to transact on the block chain, or
unilaterally determine the order in which transactions are allowed to
confirm.  Resistance to censorship is achieved by ensuring anyone and
everyone who desires has the ability to mine blocks if they so choose,
and that the probability of finding a block, and of having that block
accepted by the network and not later reorg'd out, is precisely
proportional to one's share of the hashpower, no matter how large or
small that share is.  If Alice has twice as much hashpower as Bob,
then she has twice as much say into transaction ordering as him, but
no more and no less.  And since mining is by lottery, even a small
percentage hashrate miner still has a chance to mine a block in due
time.  As the miner of a block determines the order of transactions
contained within, so it is that censorship resistance is having
control over ordering of transactions shared by all users, fairly
weighted by hashpower.

Generally on-chain settlement throughput can be increased in two ways:
by allowing blocks to be mined more frequently, or by increasing the
size of blocks.  Both would adversely affect censorship resistance by
increasing the proportion of time it takes to validate and propagate a
block relative to the time it takes find it, as during propagation and
validation other miners must either generate stale work while they
wait for validation to complete, or open themselves to the risk of
mining on top of an invalid block.  Miners don't have to make this
tradeoff when building on their own blocks (as they know them to be
valid already), which provides an advantage to larger hashrate miners
who find themselves in that lucky circumstance more often than a less
powerful miner would.  Whereas the cost of validation is linear with
the on-chain settlement throughput, the relationship of throughput
with censorship resistance is non-linear and dependent on many
factors.

There is a natural floor to block propagation times due to fundamental
speed-of-light induced message latency, which limits the effectiveness
of lowering the inter-block interval compared with directly raising
weight limits.  Therefore lowering the inter-block interval is not an
effective mechanism for achieving higher on-chain settlement capacity
as the costs grow rather quickly, and in excess of the benefits
received.  We will explore however in Section~\ref{forwardblocks}
through Section~\ref{flexcap} a mechanism by which the benefits of
increasing the per-block weight limit directly can be achieved, while
also lowering the inter-block interval from the perspective of
un-upgraded nodes, without the superlinear effect on centralization
risk.
