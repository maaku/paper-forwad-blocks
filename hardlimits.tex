While the ability to increase the block rate must be limited so as to
constrain resource utilization in validators and retain censorship
resistance, which both together provide the properties we care about
and call \emph{decentralization}, let's set this aside for just a
moment and consider just what the limits are of using time-warp to
increase settlement capacity on the compatibility chain.  After
establishing what the protocol-allowed upper limits are, we will then
look at whether these limits would ever be reasonable, and ways in
which the forward block size and compatibility block interval can be
limited so as to preserve decentralization properties that we care
about in the interim.

Because a block's timestamp is constrained to be larger than the
median timestamp of the \num{11} blocks prior, a sequence of up to
\num{6} blocks may share the same timestamp before consensus rules
require block-time to tick forward.  Thus there cannot be more than
\num{6} blocks per second in expectation before block time advances at
a rate more quickly than actual (wall clock) time, which becomes
problematic as block time is not allowed to exceed actual/wall clock
time by more than \SI{2}{\hour} at the tip.

At this maximum sustainable rate \SI{14.4}{\giga\weight} of
transactions could be processed in \num{3600} full compatibility
blocks every \SI{10}{\minute},\footnote{$600 \times 6 =$ \num{3600}
  blocks $\times \frac{\SI{4}{\mega\weight}}{\mathrm{block}} =
  \SI{14.4}{\giga\weight}$} with one compatibility block every
\SI{167}{\milli\second} in expectation.  This is an effective
\num{3600}x transaction processing rate increase and represents the
actual hard limit allowed by the bitcoin protocol, before this scaling
approach would break forward compatibility with older clients.  Should
this limit be reached, the block timestamps of the compatibility chain
would be at least two weeks old and advancing by one tick every six
blocks, with each block arriving every \sfrac{1}{6}th of a second in
expectation.  Every \num{2016} blocks, which would occur every five
and half minutes, normal difficulty adjustment would be prevented by
setting the timestamp to be approximately the present wall-clock time,
two weeks later than the \num{2015}th prior block timestamp.  The next
block that follows returns to the pattern of block timestamps
increasing as slowly as possible.

In a Section~\ref{parameters} we will evaluate how ridiculous---or
not---this real upper limit is, whether it would ever be reasonable to
allow such upper limits to be reached, and by what mechanism lower
limits could be enforced in the interim.
