Forward blocks and the compatibility chain work together to provide a
smooth transition in difficulty from the old-PoW algorithm to the new.
Achieving this difficulty transition requires a proportional
allocation of reward between old-PoW and new-PoW miners that adjusts
continuously and slowly enough over time as to prevent network
disruption:

\begin{enumerate}

  \item

    Forward-block miners are allowed a proportion $P \in [0,1]$ of the
    subsidy plus fees of the their own blocks.  The remaining $1 - P$
    portion is set aside for the compatibility block miners.

    For example, if $P = 25\%$ and the subsidy is \SI{6.25}{\bitcoin}
    and fees are \SI{1.75}{\bitcoin}, then the forward-block miner is
    allowed to put up to $0.25 * (\SI{6.25}{\bitcoin} +
    \SI{1.75}{\bitcoin}) = \SI{2}{\bitcoin}$ in the forward block's
    non-spendable coinbase outputs.

  \item

    At the point of activation $P$ is assigned a value that is quite
    small (almost zero), and the new proof of work is starts with a
    minimal difficulty requirement.  Over a sufficiently long period,
    the value $P$ is increased slowly and without discontinuity until
    it is has the approximate value $P = 1$.

  \item

    A smoothing filter controller $R(\mathrm{tip})$ takes as input the
    compatibility block headers, including locked-in forward block
    headers and their coinbase transactions, and outputs an allowed
    block reward for each compatibility block.  The content of the
    compatibility block, including its own forward block commitments,
    only affects the reward of future blocks.

\end{enumerate}

The naive solution of having the compatibility block miners simply
fill out the remaining subsidy of the forward blocks they confirm runs
into difficulty with the loosely coupled connection between the two
chains.  The inherently random process of mining blocks means that
some compatibility blocks will be responsible for confirming multiple
forward block headers whereas others will include none at all, to say
nothing of the perverse incentives this creates for reorgs.  The
solution of applying a smoothing filter transforms the naturally
fluctuating supply into a smooth, continuous reward function to
prevent adverse mining incentives, and has an added advantage of
making all portions of the block except for miner commitments and
payout destinations fully deterministic from prior-block data.

A simple controller that would work is ``$R(\mathrm{tip})$ is sum of
compatibility block portions of the block reward of locked-in forward
blocks minus payments already made, scaled by $\sfrac{1}{k}$'' for
some suitably large value of $k$, fixed in the consensus rules, that
puts added variance within the range of already existing natural
mining variance, and with some initial conditions to prevent block
reward discontinuity at the point of activation.\footnote{The actual
  scale factor used should incorporate the current warp factor $Q$, as
  defined in Section~\ref{flexcap}: $\frac{1}{k\ceil{Q}}$.
  Incorporating $Q$ as a component of the scaling factor bounds the
  increase in variance that would result from a steadily decreasing
  inter-block interval caused by an increasing aggregate weight limit
  on the forward block chain.}

While this serves as an adequate existence proof of a solution, design
of optimal controllers for this application is a topic deserving its
own dedicated treatment, and we will not do so here.  We remain open
to the possibility that better controllers exist and invite more
research in this area.  However it is worth noting some design
requirements:

\begin{enumerate}

  \item

    The sum of the controller outputs over time must equal the total
    amount made available to compatibility block miners as inputs to
    the controller.  The compatibility block miners must not earn more
    (or less) than their fair share.

  \item

    This fairness condition must hold true even if forward block
    miners coordinate to control the schedule of compatibility block
    income being registered in the forward blocks (e.g. by including
    or delaying high-fee transactions).

  \item

    This fairness condition must hold true even if compatibility block
    miners coordinate to control the booking of forward block income
    by choosing to commit to forward blocks at certain times.

  \item

    This fairness condition must not be sensitive to other factors
    under miner control (e.g. block timestamps, or block stuffing with
    miner-generated transactions).

\end{enumerate}

By providing a smooth transition from the regime of $P \approx 0$ to
$P \approx 1$ over a period of time comparable to hardware replacement
cycles, the loss of reward by old-PoW miners occurs slowly enough to
be planned for and without risk of sharp discontinuity in hashrate and
long block delays, while still achieving the goal of having the
old-PoW be trivial to maintain (e.g. difficulty \num{1}) at the end of
the transition period.

\subsection{Smoothing Out Bitcoin's Subsidy Schedule}

The coinbase payout queue dissociates the block reward paid out to
miners of both chains from the schedule of subsidy collected on the
compatibility chain and used to process the coinbase payout queue.
The reward received by the forward block miner depends on the block
that they build, and the reward received by compatibility block miners
depends on the history of forward blocks locked-in on that chain.
Neither is necessarily tied to the subsidy-dropping schedule
originally set by Satoshi---the block reward for both sets of miners
is ultimately determined by the forward block chain, which has
\textbf{no} requirement to match bitcoin's subsidy schedule:

\begin{itemize}

  \item

    If there is greater subsidy on the forward block chain than the
    compatibility chain---due to a faster subsidy schedule or less
    than expected number of compatibility blocks---then the coinbase
    payout queue grows in length and miners have to wait
    correspondingly longer to receive payment.

  \item

    If there is less subsidy on the forward block chain than the
    compatibility chain---due to a slower subsidy schedule or
    greater than expected number of compatibility blocks---then the
    coinbase payout queue may be processed completely and excess
    coinage accumulates in the carry-forward balance.

\end{itemize}

It is certainly not a problem for a carry-forward balance to exist,
even a large one.  And while a coinbase payout queue causes delays in
coin maturation, it does not otherwise cause any other issues for
miners or validators.  Because of this we have the freedom to change
the subsidy schedule for forward blocks, so long as:

\begin{enumerate}

  \item

    The cumulative subsidy never exceeds the \SI{21e6}{\bitcoin} total
    monetary base; and

  \item

    The cumulative subsidy at any given time does not exceed the
    bitcoin subsidy schedule by too large a margin as to cause
    unreasonable delays in the coinbase payment queue.

\end{enumerate}

So long as these conditions are met the coinbase payout queue may vary
in size, but will not grow to excess, beyond what can be handled with
a reasonable delay to coinbase maturation.  To prevent unreasonable
delays the subsidy curve for the forward block chain needs to
approximately match the pre-activation subsidy curve, but it need not
follow exactly the same curve as some variance or local deviation is
allowed.\footnote{Indeed, with the \SI{15}{\minute} block times
  recommended in Section~\ref{parameters}, the per-block subsidy
  schedule would need to be increased $50\%$ if nothing else.}  In
particular, it is possible to ``even out'' the subsidy curve, and
thereby remove the sharp discontinuities which presently exist, the
so-called ``halvenings.''

\begin{figure}
  \centering
  \includegraphics[width=0.5\textwidth]{subsidycurve.eps}
  \caption{Alternative Subsidy Curve}
  \label{subsidycurve}
\end{figure}

The simplest correction would be to transform the \SI{4}{\year}
constant subsidy per halving period into a linear reduction over
\SI{8}{\year} instead, resulting in a smooth subsidy curve without any
discontinuities, as shown in Figure~\ref{subsidycurve}.  Regions of
the graph for which the original/Satoshi subsidy curve exceed the
linear-interpolation curve are periods of time where the excess
subsidy on the compatibility chain would be banked in the
carry-forward balance.  That banked value would be used to pay out
miners in those regions where the original subsidy value is less than
the smoothed approximation.

In this way the dangerously incentive-incompatible ``halvening''
phenomenon can be removed from bitcoin, resulting in a continuous and
predictable subsidy drop over time.
