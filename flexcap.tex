The aggregate block limits of forward blocks are not restricted to be
the same as the aggregate limits enforced in pre-activation consensus
rules (and carried over to the compatibility chain); both the maximum
block weight and signature operation limits\footnote{Throughout this
text we will refer to the block weight limit as the defining aggregate
limit, but we mean both.  An increase or decrease of one limit in a
forward block chain would presumably be accompanied by a proportional
increase or decrease of the other limit, or the forward block chain
may forego having a separate sigop limit entirely.} can have different
values enforced in forward and compatibility blocks, so long as no
single transaction exceeds the per-block limit of the compatibility
chain.  If the forward block consistently has more transaction-weight
than can fit in a compatibility block, transactions will accumulate
for as long as this remains the case; with larger forward blocks comes
increasing pressure on the compatibility block miners to process the
transaction and coinbase payout queues.

But if the difficulty of the compatibility block chain were to be
lowered, then the same level of network hashrate would generate more
compatibility blocks within a comparable interval of time.  Under
normal circumstances this would cause the difficulty to increase at
the next adjustment up to the requisite level to
maintain \SI{599.7}{\second} expected block times.  However a flaw in
bitcoin's difficulty adjustment algorithm can be used to trigger a
difficulty lowering and used again to prevent any follow-up
re-adjustments, thereby increasing the frequency at which
compatibility blocks are found for as long as the flaw is exploited.
This flaw is known as the time-warp bug: the
\num{2016}-block difficulty adjustment interval is larger than the
\num{2015}-block window used to make the calculation,\footnotemark
allowing for the careful use of bogus timestamps at the endpoints of
this window to arbitrarily adjust difficulty in either direction up to
the maximum adjustment limit of $\pm 4$x.

\footnotetext{This is, incidentally, why bitcoin targets
  approx. \SI{599.7}{\second} block intervals rather than exactly
  $\SI{10}{\minute} = \SI{600}{\second}$:

  \begin{equation*}
    \SI{600}{\second} * \frac{2015}{2016} \approx \SI{599.7}{\second}
  \end{equation*}
}

We can combine both of these techniques of (1) increasing aggregate
limits for forward blocks, and (2) decreasing the inter-block interval
for compatibility blocks to achieve a higher transaction processing
throughput, achieving effectively the same results as direct block
weight limit increase on the original chain, but in a soft-fork
compatible way.  We will later show that while such an aggregate limit
increase raises validation costs for all nodes, large gains in
transaction processing throughput can be had without increasing other
centralization risks.

\subsection{Initial Parameters of the Forward Block Chain}

On activation of forward blocks we are given a one-time opportunity to
alter the block-level consensus rules of the forward block chain in
ways that on the compatibility chain would require a hard-fork.  We
specifically recommend two changes that permit larger transaction
processing throughput without decreasing censorship resistance:

\begin{enumerate}

  \item

    Increase the inter-block interval to a larger value than the
    original chain's \SI{10}{\minute} / \SI{599.7}{\second} target to
    reduce the ratio of network latency to the inter-block interval.
    We recommend a value of \SI{15}{\minute} / \SI{900}{\second} as a
    compromise between transaction processing throughput and utility
    (confirmation time).

  \item

    Raise the (initial) forward block maximum weight
    from \SI{4}{\mega\weight} to a recommended value
    of \SI{6}{\mega\weight}, permitting approximately the same
    transaction processing rate per unit time as the original chain.

\suspend{enumerate}

Increasing the aggregate block weight limit while simultaneously
increasing the inter-block interval is a net gain for centralization
resistance even though the on-chain settlement throughput remains the
same.  This is because the fundamental network latency is unchanged,
and the network latency portion of block propagation now constitutes a
smaller percentage of the total inter-block time.  This permits a
slightly higher forward block weight limit for the same level of
censorship resistance.

The full initial parameters of the forward block chain and their
justification are given in Section~\ref{parameters}.

\subsection{A Flexible Cap on Forward Block Weight}

In the new consensus rules on the forward block chain we now permit
under certain circumstances a larger or smaller block weight limit, at
the discretion of the new-PoW miner according to the following rules:

\begin{enumerate}

  \item

    Forward block miners are allowed to optionally *bias* their
    proof-of-work target by up to $\pm 25\%$.  This essentially makes
    their block between $25\%$ easier and $25\%$ harder to solve,
    which implicitly increases or reduces the miner's block reward
    expectation.

  \item

    Biasing the work target has an inverse-quadratic relationship with
    the adjustment made to the aggregate weight limit for the block
    relative to some baseline.  Specifically, for a bias value $x =
    \Delta T / T_0$ (permitted values of $x \in [-0.25, 0.25]$), the
    permitted block weight w is given by:

    \begin{equation} \label{rewardbiaseq}
      w(x) = w_0(2x - 4x^2)
    \end{equation}

    Equation~\eqref{rewardbiaseq} is depicted in
    Figure~\ref{rewardbiasfig}.

\suspend{enumerate}

\begin{figure}
  \centering
  \includegraphics[width=0.5\textwidth]{rewardbias.eps}
  \caption{Relationship between max block weight and difficulty bias.}
  \label{rewardbiasfig}
\end{figure}

Note that lowering the proof-of-work target, which permits slightly
larger blocks that are harder to find, is asymmetric with respect to
raising the target, which permits much smaller blocks that are easier
to find.  A non-linear equation is required for there to be a fixed
optimal value for a given transaction fee distribution.

The largest temporary block-weight increase that can be obtained is
+$25\%$, with a $+25\%$ bias to the proof-of-work target, whereas the
smallest temporary block-weight is $-75\%$ the default value, obtained
with a $-25\%$ bias.

Critically, this bias adjustment does not affect the representative
work of the block for the purpose of determining the most-work
chain. A miner cannot ``win out'' over another's block by choosing a
higher work bias.

\resume{enumerate}

  \item

    Miners provide in their blocks a declared weight, which must be
    greater than or equal to the actual weight of the block under
    current consensus rules, and less than or equal to the maximum
    weight permitted given their choice of work bias.\footnote{The
    declared weight is specified and constrained to a range rather
    than being a calculated value for reasons of forward compatibility
    with future soft-fork extensions.}

  \item

    Every $L$ forward blocks (we propose $L = 2016$, approx. \num{3}
    weeks), two things happen:

    \begin{enumerate}

      \item

        The base limit $w_0$ adjusts to the (possibly gain limited)
        average of the past $L$ declared weights.

      \item

        The new-PoW difficulty adjusts up or down as necessary to
        maintain the \SI{15}{\minute} inter-block
        interval.\footnote{If bitcoin's difficulty adjustment
        algorithm is used, the time-warp bug should be fixed for
        forward blocks.}

    \end{enumerate}

\end{enumerate}

With the above rules, the forward block chain is able to provide burst
transaction processing capacity as needed, and automatically adjusts
its base limits in response to real demand.  Miners are prevented from
unilaterally increasing aggregate limits due to the prohibitive costs
they would suffer directly as a result.

Should the forward block chain be subject to an aggregate sigop limit,
it too could be raised or lowered by a proportional amount.  However
it is the opinion of this author that advances in validation speed and
the elimination of quadratic costs have rendered the sigop limit
mostly redundant, and as such the forward block chain should not be
subject to an aggregate sigop limit at all.

\subsection{Processing Larger Blocks on the Compatibility Chain via a Coordinated Time-Warp}

Now that we've established how the forward block chain manages its
aggregate weight limit, we can look at how the compatibility chain
adjusts its inter-block interval to achieve a matching transaction
processing rate, through coordinated exploitation of the time-warp
bug.

Upon activation, the we grant the compatibility chain new consensus
rules governing its block timestamps:

\begin{enumerate}

  \item

    If a compatibility block is \textbf{not} the last block in a
    difficulty adjustment period (the block height is non-zero
    modulo \num{2016}):

    \begin{enumerate}

      \item

        If after constructing the block the transaction queue is
        empty, no additional rules apply.

      \item

        Otherwise, if the next transaction in the queue is non-final
        for reasons of a time-based lock-time or sequence-lock, the
        block timestamp is set to the minimum value necessary to
        satisfy that lock condition.

      \item

        Otherwise, the block's timestamp \textbf{must} be set to its
        minimum allowed value, which is one more than the median of
        the \num{11} blocks prior.

    \end{enumerate}

\suspend{enumerate}

The implication of this rule is that so long as compatibility blocks
are full, the timestamps of compatibility block headers advance no
more quickly than one second every six blocks, until such time as the
queue is emptied or the next item in the queue requires a higher
timestamp value.  As the flexible weight limit of the forward block
chain is expanded, the transaction queue will fill up and the
difficulty of the comptibility chain will need to be lowered to keep
up.  Making sure timestamps advance as slowly as possible ensures this
can happen.

The mechanism for causing the difficulty adjustment is a bit more
complicated:

\resume{enumerate}

  \item

    The compatibility block miner makes a commitment alongside forward
    block headers to the aggregate declared weight of the forward
    block chain up to that point.\footnote{If sharding is used (see
    Section~\ref{sharding}), there is one aggregate value per shard
    chain and the number considered here is the sum of the aggregate
    locked-in weight across all shards.}  For the header to transition
    to locked-in status, the commitment must match the same values
    reported in the forward block chain, or else the compatibility
    block containing the $N$th forward block commitment is invalid.

  \item

    Since the {\tt block.nTime} field now has its value subject to
    additional rules when blocks are full, the compatibility block
    miner commits to the wall-clock time in the {\tt nLockTime} field
    of its coinbase transaction.\footnote{Implementation detail: the
    actual commitment would be to {\tt LOCKTIME\_THRESHOLD} plus the
    zero-clamped difference between the wall-clock time and the median
    of the past \num{11} block timestamps, or some other calculation
    which ensures the value is always larger than {\tt
    LOCKTIME\_THRESHOLD} but less than the median past \num{11} block
    timestamps.  The above rules should be interpreted accordingly.
    Later references to this field refer to the actual UNIX timestamp,
    not the {\tt LOCKTIME\_THRESHOLD}-relative encoded value.}  This
    commitment is subject to roughly the same rules that have always
    applied to the block timestamp:

    \begin{enumerate}

      \item

        The coinbase {\tt nLockTime} must be greater than the median
        of the past \num{23} coinbase {\tt nLockTime}'s.\footnotemark

      \item

        The coinbase {\tt nLockTime} commitment must be no more than
        two hours ahead of the current wall-clock time.

    \end{enumerate}

    \footnotetext{The median-time-past window is doubled to prevent
    ratcheting forward of the timestamp as a result of normal variance
    once compatibility chain scaling limits are reached.}

  \item

    If a compatibility block \textbf{is} the last block in a
    difficulty adjustment period (the block height is zero
    modulo \num{2016}), then the block's timestamp is required to be
    set according to the following procedure:

    \begin{enumerate}

      \item

        A ratio $Q$, the \emph{warp factor}, is calculated as the
        ratio of forward block utilization\footnotemark to the
        original chain's settlement capacity:

        \begin{equation}
          Q = \frac{w}{\SI{6}{\mega\weight}}
        \end{equation}

        \footnotetext{Aggregate across all shards, if sharding is used:
          \begin{equation*}
            Q = \frac{\sum_{i=1}^{28}w_i}{\SI{6}{\mega\weight}}
          \end{equation*}
        }

        The integer $\ceil{Q}$ is the same value, rounded up.

      \item

        Using the last $2016 \times \ceil{Q}$ compatibility blocks'
        wall-clock timestamp information (the coinbase {\tt nLockTime}
        fields, not the block header timestamp), a calculation is made
        of which proof-of-work target over that approximately two week
        period would have yielded $Q$ blocks in expectation
        every \SI{600}{\second}, the necessary rate in order to
        process forward blocks at their present size.

      \item

        The original/Satoshi difficulty adjustment formula is applied
        in reverse to calculate the block timestamp which would cause
        in an adjustment to the proof-of-work target calculated above.

      \item

        The block timestamp is constrained by clamping to be no
        smaller than its minimum allowed value of one more than
        the \num{11} block median time past from block headers, and no
        greater than \SI{14400}{\second} (\SI{4}{\hour}) beyond the
        last block's coinbase committed wall-clock
        value.

    \end{enumerate}

\end{enumerate}

Together these rules cause the compatibility chain's block rate to
adapt to whatever the forward block chain's transaction processing
throughput is, while still remaining responsive to changes in its own
network hash rate, while also preventing compatibility block
timestamps from being set so high as to cause a network split due to a
future-dated block.

We've now covered how (1) the forward block chain can grow (or shrink)
in response to real demand; and (2) how the time-warp bug can be
exploited to adjust the inter-block interval of the compatibility
chain to make sure every forward-block transaction reaches every node.
We look in Section~\ref{hardlimits} at what the hard limits are to
scaling transaction processing capacity via this approach.

\subsection{The Centralization Risks of a Flexible Weight Limit}

The compatibility chain is restricted to increasing its transaction
processing throughput by decreasing its inter-block interval only,
though exploiting the time-warp bug.  This increases validation costs
for everyone, and adds centralizing pressure for compatibility miners.
Furthermore, decreasing the compatibility miner's share of the block
reward as part of the proof-of-work transition provides additional
centralizing pressure so long as the distribution of cheap electricity
and efficient mining hardware remains unequal: as lower block reward
reduces mining profits for everyone, those with higher costs are
driven out of business first, and what remains is an oligarchy of
miners with access to the best hardware, the cheapest electricity,
and/or the largest subsidies.

With the forward block chain we have a choice of changing the block
weight limit, adjusting the inter-block interval, or both.  Increasing
forward block weight limit is preferable to decreasing the inter-block
rate because of the non-linear relationship between inter-block time
and centralization risk: some risks factors relate to block
propagation delays caused by the speed of light, and these risks are
independent of the size of a block.  We even go so far as to recommend
lengthening the inter-block time on the forward block chain in order
to achieve a small but important boost in censorship resistance.

Considered by itself, indirectly increasing transaction processing
throughput on the compatibility chain by shortening the inter-block
interval proportionally increases the cost of validation, but
does \textbf{not} affect censorship resistance.  This is because the
ordering of transactions is fully determined by the forward blocks,
and so with regard to censorship the distribution of miners on the
compatibility chain doesn't matter so long as the forward block chain
makes progress independent of the compatibility chain.

More frequent compatibility blocks would be filled by larger forward
blocks, however, and increasing the forward block size both increases
the cost of validation for full nodes, and harms censorship resistance
by disproportionately favoring large hashrate new-PoW miners.  In
Section~\ref{sharding} we examine a solution that allows us to make
more smaller, more frequent forward blocks while still maintaining the
benefits of a long inter-block interval.  But first we look at just
how much scaling advantage we can extract out of the time-warp bug on
the compatibility chain.
