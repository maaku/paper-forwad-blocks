The bitcoin block chain has per-block aggregate limits that prevent
the on-chain settlement capacity from increasing beyond fixed levels
of \SI{4}{\mega\weight} per \SI{10}{\min} in expectation.  While
seggregated witness achieved a \numrange{2}{3.6}x improvement in
settlement capacity by removing witness data from the area of the
block subject to classical limits, this was a one-time gain: there are
no other large areas of the block that can be safely segregated into
their own data structures.  Conventional wisdom is that on-chain
settlement capacity of bitcoin and bitcoin-derived systems can only be
increased by one of four different mechanisms.  Taken individually,
each approach comes with significant tradeoffs:

\begin{enumerate}

  \item

    \emph{Hard fork} change the consensus rules in a way that is not
    forwards compatible with old clients, e.g. simply increasing the
    weight limit, thereby requiring \textbf{all} nodes to upgrade by
    some flag day.  While conceptually simple, the flag day
    requirement violates many of the underlying principles of
    decentralization from which bitcoin-like systems derive their
    value.

  \item

    \emph{Soft fork} validation rules for segregating entire
    transactions from the classically visible block structure, in the
    form of a so-called \emph{extension block} which contains extra
    transactions beyond the classical limit.  While being a soft-fork,
    this category of solutions has the disadvantage of opaquely hiding
    the spend graph from un-upgraded nodes: in contrast to segregated
    witness which merely hid authorization information, an extension
    block approach will necessarily obscure entire transactions,
    including the inputs they spend and the outputs they generate from
    any un-upgraded client.  Some proposals even involve forcing empty
    non-extension blocks for the purpose of coercively shutting down
    the legacy network, forcing an upgrade via denial of service.
    Again, this violates the principles of decentralization underlying
    bitcoin-like systems.

  \item

    \emph{Soft fork} validation rules for transferring value between
    sidechains enables scaling to happen on a separate network using
    tokens ultimately backed by original currency from the source
    chain, e.g. bitcoin.  While sidechains are interesting as a way of
    locking value for use in new and novel off-chain ledgers providing
    features which are either immature or will never make it to the
    mainchain, as a mechanism for scaling it doesn't compare well.
    Either SPV trust assumptions are required for the value transfer
    authorization, or full validation of the other block is required
    and the sidechain mechanism becomes essentially an overly
    complicated implementation of extension blocks without the
    empty-block coercion, and inherits all the same negative tradeoffs
    regarding ledger opacity to un-upgraded nodes.

  \item

    Manipulation of timestamps affecting the difficulty-adjustment
    algorithm to make blocks more frequent and thereby indirectly
    achieve similar results to a block size increase.  Known as the
    \emph{time-warp attack}, this exploit has been used to play out
    subsidy in various altcoins going back as far as at least 2011.
    It has been noticed by various people that coordinating an
    exploitation of this flaw to lower inter-block intervals would
    effectively achieve similar results as a block size increase in
    terms of on-chain settlement throughput.  However due to
    fundamental limitations in latency-induced message propagation
    times, and the resulting consequences for centralization risk,
    comparatively very little gain can be had from using a time-warp
    exploit before the risks become unacceptable.

\end{enumerate}

While each of these approaches individually have unacceptable
trade-offs, it turns out, remarkably, that combining them all together
``cancels out'' most of the bad tradeoffs wile retaining the combined
benefits.  The resulting scheme, held together by a novel new
mechanism we call \emph{forward blocks}, is actually less complicated
than one might think of a ``everything and the kitchen sink''
proposal.

Additionally, it was previously considered that upgrading the
proof-of-work function was not possible in an incentive compatible
way, and therefore necessarily a hard-fork alteration of the consensus
rules.  However implementation of forward blocks enables, and in fact
requires a soft-fork proof-of-work change\footnote{The ``new''
  proof-of-work could be a variation of merged mining, however, should
  retiring existing mining hardware be deemed an anti-goal.} as
natural consequence of its construction.  In fact this insight into
accomplishing soft-fork upgrrades of the proof-of-work function is the
driving idea behind the forward blocks construction, and where we
begin the exposition in Section~\ref{dualpow}, after a brief
discussion of centralization risks in
Section~\ref{centralizationrisk}.

Taking the concept of a soft-fork proof-of-work change to its logical
conclusion, we develop in Sections~\ref{forwardblocks},
\ref{coinbasepayoutqueue}, and~\ref{smoothdifficulty} a two-chain
structure where the \emph{forward block chain} establishes a
consistent transaction ordering reflected in the \emph{compatibility
  block chain} after a loosly coupled process of state
synchronization.

To actually increase on-chain settlement throughput,
Section~\ref{flexcap} describes the approach of directly increasing
aggrergate limits used on the forward block chain, synchronized with
time-warps to lower the inter-block interval on the compatibility
chain, all constrained by a demand-responsive and growth-constrained
flexible weight cap.

To safely reach scaling limits, Section~\ref{sharding} describes how
elements of the sidechain value peg mechanism are used to enable
\emph{sharding} across multiple forward block chains.  Once fully
utilized, an approx. \num{30}x reduction in latency-induced
centralization risk can be achieved.

We then present and justify the set of initial parameters and
long-term growth constraints in Section~\ref{parameters}.

Finally, it turns out the mechanism developed for transferring value
between shards is generally applicable as a mechanism for handling
coinbase maturation requirements and tranferring value between
segregated ledgers.  We briefly examine in
Section~\ref{extensionoutputs} how it can be used to provide a
rebateable fee market for consensus fee detection as well as handle
the transfer of explicit value between confidential transactions /
mimblewimble, unlinkable anonymous spends, and sidechain ledgers.
