There's a one-time win for decentralization that can be had
simultaneous with the deployment of forward blocks, and that comes
from \emph{sharding}.  By supporting multiple forward block chains,
or \emph{shards}, higher total transactin processing rates can be
achieved for the same level of latency-induced centralization risk.

The technique is to explicitly segment outputs into disjoint UTXO sets
for each \emph{shard}, and to require that all inputs to a transaction
draw from a single shard.  Outputs may explicitly name a different
shard that they belong to, but at the cost of a delayed transfer and
maturation process achieving loose coupling of state between shards.
The compatibility block chain tracks the block headers of all shards,
and add transactions from each shard to the processing queue as the
shard's forward blocks achieve $N$ confirmations.  When a transaction
in one shard has an output destined for a different shard, the
compatibility block chain adds that destination and amount to the
coinbase payout queue and claims the output as part of the
carry-forward balance.  Once paid out and mature, the coinbase payout
enters the UTXO set of the destination shard.  The following rules
define how outputs enter the UTXO set of the forward block shards,
becoming spendable:

\begin{enumerate}
  \item
    Each shard constitutes a disjoint ledger of value, with a UTXO set
    that does not intersect with the UTXO set of any other shard.  As
    a consequence, transactions must source all their inputs from the
    same shard.

  \item
    All legacy outputs and non-prefixed segwit outputs of a
    transaction belong to the shard that transaction was confirmed in
    (the same shard it sourced its inputs from), and are immediately
    spendable by later transactions of the same shard.

  \item
    A prefixed native segwit output in a user transaction differs from
    a native segwit output in having a prefix byte indicating its
    destination shard.  Prefixed native segwit outputs do \textbf{not}
    enter the UTXO set of any forward block shard.

  \item
    As transactions are added to the transaction processing queue of
    the compatibility chain, any prefixed native segwit outputs in the
    transactions of that forward block are added to the coinbase
    payout queue.

  \item
    When a transaction is processed from the queue and included in a
    compatibility block, the prefixed segwit output is spent by a
    miner-generated transaction within the same block, and its value
    is added to the carry-forward balance used to process items from
    the coinbase payout queue.

  \item
    A prefixed native segwit output in the coinbase transaction of a
    compatibility block becomes spendable in the shard specified by
    its prefix, and that shard only, through the normal process of
    maturation.  Any legacy or non-prefixed native segwit coinbase
    outputs mature into the default, first shard.
\end{enumerate}

As a prefix we recommend using a single-byte script opcode, for
efficiency.  There are a total of 28 single-byte script opcodes which
may be safely executed at the beginning of a scriptPubKey context:
{\tt 1NEGATE}, {\tt FALSE}, {\tt TRUE}, the remaining \num{15} small
number pushes (\numrange{2}{16}), {\tt NOP}, {\tt NOP1}, {\tt NOP4}
through {\tt NOP10}, and {\tt DEPTH}.  {\tt CODESEPARATOR} could be
used but is excluded from this list due to potential infrastructure
screwups relating to its other, complicated semantics.
Table~\ref{shardprefix} enumerates these shard prefixes in order.

\begin{table} \center
  \caption{Shard Prefixes}
  \label{shardprefix}
  \begin{tabular}{r l l}
    \toprule
    Shard & Prefix & Opcode\\
    \midrule
    1   &  $\mathtt{0x00}$  &  {\tt FALSE}\\
    2   &  $\mathtt{0x4f}$  &  {\tt 1NEGATE}\\
    3   &  $\mathtt{0x51}$  &  {\tt TRUE}\\
    4   &  $\mathtt{0x52}$  &  \num{2}\\
    5   &  $\mathtt{0x53}$  &  \num{3}\\
    6   &  $\mathtt{0x54}$  &  \num{4}\\
    7   &  $\mathtt{0x55}$  &  \num{5}\\
    8   &  $\mathtt{0x56}$  &  \num{6}\\
    9   &  $\mathtt{0x57}$  &  \num{7}\\
    10  &  $\mathtt{0x58}$  &  \num{8}\\
    11  &  $\mathtt{0x59}$  &  \num{9}\\
    12  &  $\mathtt{0x5a}$  &  \num{10}\\
    13  &  $\mathtt{0x5b}$  &  \num{11}\\
    14  &  $\mathtt{0x5c}$  &  \num{12}\\
    15  &  $\mathtt{0x5d}$  &  \num{13}\\
    16  &  $\mathtt{0x5e}$  &  \num{14}\\
    17  &  $\mathtt{0x5f}$  &  \num{15}\\
    18  &  $\mathtt{0x60}$  &  \num{16}\\
    19  &  $\mathtt{0x61}$  &  {\tt NOP}\\
    20  &  $\mathtt{0x74}$  &  {\tt DEPTH}\\
    21  &  $\mathtt{0xb0}$  &  {\tt NOP1}\\
    22  &  $\mathtt{0xb3}$  &  {\tt NOP4}\footnotemark\\
    23  &  $\mathtt{0xb4}$  &  {\tt NOP5}\\
    24  &  $\mathtt{0xb5}$  &  {\tt NOP6}\\
    25  &  $\mathtt{0xb6}$  &  {\tt NOP7}\\
    26  &  $\mathtt{0xb7}$  &  {\tt NOP8}\\
    27  &  $\mathtt{0xb8}$  &  {\tt NOP9}\\
    28  &  $\mathtt{0xb9}$  &  {\tt NOP10}\\
    \bottomrule
  \end{tabular}
\end{table}

\footnotetext{In Tradecraft, {\tt NOP2} and {\tt NOP3} are also used
  for a total of \num{30} shards.}

Any user transaction output consisting of one of these prefixes
followed by normal native segwit output script (a \numrange{0}{16}
small integer push followed by a small data push) is called a prefixed
native segwit output, and follows the process of coinbase maturation
detailed above.  A coinbase payout of this form enters the UTXO set of
the specified shard upon maturation, and can be spent within that
shard exactly as if it were a native segwit script without the prefix.

Without complicating the prefix scheme further, the number of shards
is limited by the number of prefix bytes available, which is a
historical artifact of bitcoin script.  As it happens however, the
number of one-byte prefixes which can be safely executed at the
beginning of a push-only script does align rather closely with the
number of shards which can be safely allowed given other design
constraints.

\subsection{The Impact of Sharding on Centralization Risks}

The maximum forward block size limits discussed previously
of \SI{14.4}{\giga\weight} / \SI{10}{\minute} are far in excess of
what could be supported by any presently imaginable network
architecture: the centralization pressures would destroy the utility
of the network, even with generous assumptions about future bandwidth
and latency numbers.

But by allowing some complexity to be pushed onto the user in the form
of sharding, we can make full utilization of the available capacity of
a time-warped compatibility block chain without growing individual
forward block shard chains to be in excess of the upper limits of
previous conservative block size increase proposals, such as BIP-103.

Increasing the effective transaction processing rate through sharding
proportionally increases the cost of validation for full nodes.  For
lightweight clients, resources are only increased in proportion to the
number of shards they have outputs on or are watching for payments.

However increasing the effective transaction processing rate through
sharding does not increase latency-dependent censorship risks, at
least while the number of shards remains within reasonable limits.
Each shard is its own long-interval forward block chain operating
independently of the other shards, with a separately salted proof of
work and a loosely-coupled value transfer process across shards.  The
separate and salted proof-of-work requirement ensures that there is no
correlation between when blocks are found on different shards.  The
coinbase maturation period ensures that the state of one shard has no
low-latency dependencies on the state of any other shard.  Together
these prevent the introduction of centralizing mining incentives from
the transaction weight of other shards, preventing a decrease in
censorship resistance, but only so long as the time it takes to
validate a shard remains significantly less than the expected time
between any shard block being found.  \num{28} shards, with each shard
having a \SI{15}{\minute} inter-block time would have shards arriving
every $\sim\SI{32}{\second}$ in expectation on bitcoin, every
\SI{30}{\second} on Tradecraft.  On the assumption that individual blocks
should not take more than a handful of seconds per block to validate
in the worst case, this provides ample room even in the face of mining
variance, the flex cap, and adversarial blocks to ensure that multiple
blocks arriving simultaneously (and therefore impacting each others'
validation time) is not the norm.
