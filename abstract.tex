Segregated witness achieved a \numrange{2}{3.6}x capacity increase for
on-chain settlement by moving script and signature data out of the
area constrained by the original \SI{1}{MB} block size limit.
Unfortunately this approach could only be used to achieve a one-time
increase as there is no remaining unsigned transaction data of
significant size that could be moved out of the transaction structure.

The previously proposed concept of soft-fork extension blocks involves
entire transactions being moved into segregated data structures, which
allows for limitless scaling at the cost of breaking the transaction
graph.  Alternatively simply removing or replacing block-size and
other aggregate limits preserves the transaction graph but at the cost
of a hard-fork cutoff event, and the significant transition risks and
centralizing pressure that comes with any scheduled hard-fork event.

We introduce the concept of \emph{forward blocks} which combine the
best of both approaches enabling aggregate block limits on size and
number of signature operations to be circumvented as a soft-fork,
while preserving the property that the spend graph, including all
transactions and their witnesses, is seen by legacy clients.
Effective block sizes of up to \SI{14.336}{\giga\weight} can be
achieved in this way, permitting scaling up on-chain/settlement
transaction volume to \num{3584}x current levels, if permitted by the
new consensus rules.

The construction of forward blocks also makes possible changing the
proof-of-work algorithm in a soft-fork compatible way, either
simultaneously with dropping the aggregate limits or as a later
re-application of the concept in nested forward blocks.  Other
benefits of the implementation approach taken include the introduction
of explicit sharding for better scaling properties, rebateable fee
market for consensus fee detection, and smoothing out drops in miner
subsidy.  It also provides the necessary prerequisite protocol pieces
for confidential transactions, mimblewimble, unlinkable anonymous
spends, and sidechains.
